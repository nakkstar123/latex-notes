\documentclass{article}

\usepackage{fancyhdr}
\usepackage{extramarks}
\usepackage{amsmath, amssymb}
\usepackage{amsthm}
\usepackage{amsfonts}
\usepackage{tikz}
\usepackage[plain]{algorithm}
\usepackage{algpseudocode}

\newcommand\N{\ensuremath{\mathbb{N}}}
\newcommand\R{\ensuremath{\mathbb{R}}}
\newcommand\Z{\ensuremath{\mathbb{Z}}}
\renewcommand\O{\ensuremath{\emptyset}}
\newcommand\Q{\ensuremath{\mathbb{Q}}}
\newcommand\C{\ensuremath{\mathbb{C}}}
\newcommand\Ha{\ensuremath{\mathbb{H}}}
\newcommand\cont{\Rightarrow\!\Leftarrow}
\newcommand\tf{\therefore}

\usetikzlibrary{automata,positioning}

%
% Basic Document Settings
%

\topmargin=-0.45in
\evensidemargin=0in
\oddsidemargin=0in
\textwidth=6.5in
\textheight=9.0in
\headsep=0.25in

\linespread{1.1}

\pagestyle{fancy}
\lhead{\hmwkAuthorName}
\chead{\hmwkClass\ : \hmwkTitle}
\rhead{\firstxmark}
\lfoot{\lastxmark}
\cfoot{\thepage}

\renewcommand\headrulewidth{0.4pt}
\renewcommand\footrulewidth{0.4pt}

\setlength\parindent{0pt}

%
% Create Problem Sections
%

\newcommand{\enterProblemHeader}[1]{
    \nobreak\extramarks{}{Problem \arabic{#1} continued on next page\ldots}\nobreak{}
    \nobreak\extramarks{Problem \arabic{#1} (continued)}{Problem \arabic{#1} continued on next page\ldots}\nobreak{}
}

\newcommand{\exitProblemHeader}[1]{
    \nobreak\extramarks{Problem \arabic{#1} (continued)}{Problem \arabic{#1} continued on next page\ldots}\nobreak{}
    \stepcounter{#1}
    \nobreak\extramarks{Problem \arabic{#1}}{}\nobreak{}
}

\setcounter{secnumdepth}{0}
\newcounter{partCounter}
\newcounter{homeworkProblemCounter}
\setcounter{homeworkProblemCounter}{1}
\nobreak\extramarks{Problem \arabic{homeworkProblemCounter}}{}\nobreak{}

%
% Homework Problem Environment
%
% This environment takes an optional argument. When given, it will adjust the
% problem counter. This is useful for when the problems given for your
% assignment aren't sequential. See the last 3 problems of this template for an
% example.
%
\newenvironment{homeworkProblem}[1][-1]{
    \ifnum#1>0
        \setcounter{homeworkProblemCounter}{#1}
    \fi
    \section{Problem \arabic{homeworkProblemCounter}}
    \setcounter{partCounter}{1}
    \enterProblemHeader{homeworkProblemCounter}
}{
    \exitProblemHeader{homeworkProblemCounter}
}

%
% Homework Details
%   - Title
%   - Due date
%   - Class
%   - Section/Time
%   - Instructor
%   - Author
%

\newcommand{\hmwkTitle}{Homework\ \#1}
\newcommand{\hmwkDueDate}{January 18, 2022}
\newcommand{\hmwkClass}{Math 131AH}
\newcommand{\hmwkClassTime}{131AH}
\newcommand{\hmwkClassInstructor}{Professor Marek Biskup}
\newcommand{\hmwkAuthorName}{\textbf{Nakul Khambhati}}

%
% Title Page
%

\title{
    \vspace{2in}
    \textmd{\textbf{\hmwkClass:\ \hmwkTitle}}\\
    \normalsize\vspace{0.1in}\small{Due\ on\ \hmwkDueDate}\\
    \vspace{0.1in}\large{\textit{\hmwkClassInstructor}}
    \vspace{3in}
}

\author{\hmwkAuthorName}
\date{}

\renewcommand{\part}[1]{\textbf{\large Part \Alph{partCounter}}\stepcounter{partCounter}\\}

%
% Various Helper Commands
%

% Useful for algorithms
\newcommand{\alg}[1]{\textsc{\bfseries \footnotesize #1}}

% For derivatives
\newcommand{\deriv}[1]{\frac{\mathrm{d}}{\mathrm{d}x} (#1)}

% For partial derivatives
\newcommand{\pderiv}[2]{\frac{\partial}{\partial #1} (#2)}

% Integral dx
\newcommand{\dx}{\mathrm{d}x}

% Alias for the Solution section header
\newcommand{\solution}{\textbf{\large Solution}}

% Probability commands: Expectation, Variance, Covariance, Bias
\newcommand{\E}{\mathrm{E}}
\newcommand{\Var}{\mathrm{Var}}
\newcommand{\Cov}{\mathrm{Cov}}
\newcommand{\Bias}{\mathrm{Bias}}

\begin{document}

\maketitle

\pagebreak

\begin{homeworkProblem}
We will construct the truth tables by evaluating intermediate expressions. For conciseness, we will abbreviate \texttt{TRUE} as \(1\) and \texttt{FALSE} as \(0\).
\begin{enumerate}
    \item \((P \lor Q) \land \lnot (P \land Q)\)
    \begin{table}[h]
        \qquad
        \begin{tabular}{|c|c|c|c|c|}
        \hline
        $P$ & $Q$ & $\lnot (P \land Q)$ & $P \lor Q$ &  \((P \lor Q) \land \lnot (P \land Q)\) \\ \hline
        0 & 0 & 1             & 0      & 0 \\ \hline
        0 & 1 & 1             & 1      & 1 \\ \hline
        1 & 0 & 1             & 1      & 1 \\ \hline
        1 & 1 & 0             & 1      & 0 \\ \hline
        \end{tabular}
    \end{table}
    \item $P \Rightarrow (Q \Rightarrow \lnot P)$
    \begin{table}[h]
        \qquad
        \begin{tabular}{|c|c|c|c|}
        \hline
        $P$ & $Q$ & $Q \Rightarrow \lnot P$ & $P \Rightarrow (Q \Rightarrow \lnot P)$ \\ \hline
        0          & 0          & 1            & 1            \\ \hline
        0          & 1          & 1            & 1            \\ \hline
        1          & 0          & 1            & 1            \\ \hline
        1          & 1          & 0            & 0            \\ \hline
        \end{tabular}
        \end{table}
\end{enumerate}

We can verify this by checking the truth table of $(P \land \lnot Q) \lor (P \land Q)$

\begin{table}[h]
    \begin{tabular}{|c|c|c|c|c|}
    \hline
    $P$ & $Q$ & $P \land \lnot Q$ & $P \land Q$ &  $(P \land \lnot Q) \lor (P \land Q)$ \\ \hline
    0 & 0 & 0             & 0      & 0 \\ \hline
    0 & 1 & 0             & 0      & 0 \\ \hline
    1 & 0 & 1             & 0      & 1 \\ \hline
    1 & 1 & 0             & 1      & 1 \\ \hline
    \end{tabular}
\end{table}

Since $P$ and $(P \land \lnot Q) \lor (P \land Q)$ have the same truth values, we conclude that $P \iff (P \land \lnot Q) \lor (P \land Q)$ and the expression is a \texttt{TAUTOLOGY}.
 
\end{homeworkProblem}

\begin{homeworkProblem}
    First, we define the proposition $m | n$ as $ \exists k \in \Z: n = km$. We now transcribe the english sentences to propositional logic:
    \begin{enumerate}
        \item $\forall n \in \N: 3|n \implies (7|n \implies 2|n)$
        \item $(\exists n \in \N: 6|n \land 4|n)\land(\exists m \in \N :  6|m \land \lnot (4|m))$
        \item ($\forall n \in \N: (6|n \implies 5|n)\implies 20|n) \land (\exists m\in \N: 6|m \land \lnot(5|m))$
        \item $\exists n \in \N : 3|n \land 7|n \land \lnot (2|n)$
        \item $(\forall n \in \N: \lnot (6|n) \lor \lnot(4|n))\lor (\forall m \in \N \lnot (6|m)\lor (4|m))$
        \item $(\exists n \in \N:(6|n \implies 5|n) \land \lnot (20|n) )\lor (\forall m\in \N: \lnot(6|m) \lor 5|m)$
    \end{enumerate}
\end{homeworkProblem}

\begin{homeworkProblem}
    We are working within the universal set $\R$ of real numbers.
    \begin{enumerate}
        \item $\forall A\subset \R (\exists x \in A: (\forall y\in A: y = x \iff y^2 = 1))$
        \item \(\forall A \subset \R (\exists x \in A: (\forall y \in A:(y\neq x \Rightarrow y < x)))\)
        \item $\forall x \in \R \; \exists A \subset \R: A \neq \O \land x \notin A$
    \end{enumerate}
\end{homeworkProblem}

\begin{homeworkProblem}
    We are asked to consider the relation \(A\subset B := (\forall x \in A: x \in B)\)
    \begin{enumerate}
        \item[(reflexive)] It is clear that \(\forall x \in A: x\in A\) so \(A\subset A\).
        \item[(antisymmetric)] Let \(A\subset B\) and \(B\subset A\). Then \(\forall x \in A: x \in B\) and \(\forall y\in B: y\in A\) so \(\forall x \in C: x \in A \iff x \in B\), therefore \(A=B\).  
        \item[(transitive)] Let \(A\subset B\) and \(B\subset D\). Let \(x\in A\). Therefore, \(x\in B\), so \(x\in D\). Since $x$ was arbitrarily chosen, we have \(\forall x \in A: x \in D\) so \(A\subset D\).
    \end{enumerate}

    This proves that the relation is a partial order. 
\end{homeworkProblem}

\begin{homeworkProblem}
   

\begin{enumerate}
    \item[(a)] We are asked to show that \(\bigcup\limits_{\alpha\in I}A_{\alpha}^c = (\bigcap\limits_{\alpha \in I}A_{\alpha})^c\).
    \begin{proof}
        \(x \ \in \bigcup\limits_{\alpha\in I}A_{\alpha}^c \iff \exists i\in I: x \in Y\setminus A_i \iff \exists i \in I: (x\in Y \land x\notin A_i) \iff x\in Y \land x\notin \bigcap\limits_{\alpha \in I}A_{\alpha}\\ \iff x \in (\bigcap\limits_{\alpha \in I}A_{\alpha})^c\) i.e. \(x \in Y \setminus \bigcap\limits_{\alpha\in I}A_{\alpha}\). Since \(x\in \texttt{LHS} \iff x \in \texttt{RHS}\), we say that \texttt{LHS = RHS}. 

        Alternatively, we can describe both sets using propositional logic:

        \texttt{LHS} = \(\left\{ x\in Y: (\exists \alpha \in I: x \notin A_{\alpha}) \right\}\)

        \texttt{RHS} = \(\left\{ x \in Y: \lnot (\forall \alpha \in I: x \in A_{\alpha}) \right\} = \left\{ x\in Y : (\exists \alpha\in I:x\notin A_{\alpha} ) \right\}\)

        Clearly, \texttt{LHS = RHS}.
    \end{proof}
    \item[(b)] We are asked to show that \(\bigcap\limits_{\alpha\in I}A_{\alpha}^c = (\bigcup\limits_{\alpha \in I}A_{\alpha})^c\).
    \begin{proof}
            \(x \ \in \bigcap\limits_{\alpha\in I}A_{\alpha}^c  \iff \forall i\in I: x \in Y\setminus A_i \iff \forall i\in I: x\in Y \land x\notin A_i \\\iff x\in Y \land x\notin \bigcup\limits_{\alpha\in I}A_{\alpha}\iff  x \in (\bigcup\limits_{\alpha\in I}A_{\alpha})^c\). Since \(x\in \texttt{LHS} \iff x \in \texttt{RHS}\), we say that \texttt{LHS = RHS}. 

            Again, we can also describe both sets using propositional logic:

            \texttt{LHS} = \(\left\{ x\in Y: (\forall \alpha \in I: x \notin A_{\alpha}) \right\}\)

            \texttt{RHS} = \(\left\{ x \in Y: \lnot (\exists \alpha \in I: x \in A_{\alpha}) \right\} = \left\{ x\in Y : (\forall \alpha\in I:x\notin A_{\alpha} ) \right\}\)

            Therefore, \texttt{LHS = RHS}.

         
    \end{proof}


\end{enumerate}



   
\end{homeworkProblem}

\begin{homeworkProblem}
    We define \([x]:= \left\{ y\in A: x \sim y \right\}\) and we have to prove that \(\forall x,y \in A: [x] = [y] \lor [x]\cap[y] = \O\).
    \begin{proof}
        Assume that \([x]\cap [y] \neq \O\) i.e. \(\exists z\in A: z\in [x] \land z \in [y]\). Then, by definition, \(x \sim z\) and \(y \sim z\). 
        By symmetry, \(z\sim y\) and by transitivity, \(x\sim y\). Then \(x \in [y]\). Let \(w\in [x]\) i.e. \(x\sim w\). Again, by symmetry and transitivity, we can show \(y \sim w\) so \(w\in [y]\). This implies that \([x]\subset [y]\). 
        Similarly, we can argue that \([y]\subset [x]\) so \([x] = [y]\). We have shown that if \([x]\) and \([y]\) are not disjoint then \([x]=[y]\). So, we have proved the claim that \([x] = [y] \lor [x]\cap[y] = \O\). Since \(x,y \in A\) were 
        arbitrarily chosen, this holds \(\forall x,y \in A\).
    \end{proof}
\end{homeworkProblem}

\begin{homeworkProblem}

    \begin{proof}
        It is clear that \(x = x' \land y = y' \implies \left\{x, \left\{ x,y \right\} \right\} = \left\{x', \left\{ x',y' \right\} \right\} \iff (x,y) = (x', y')\).

    Now assume that \((x,y) = (x', y') \iff \left\{x, \left\{ x,y \right\} \right\} = \left\{x', \left\{ x',y' \right\} \right\}\). 
    Then, by size considerations, \(x = x' \land \left\{ x,y \right\} = \left\{ x',y' \right\}\) so \(y = y'\).
    \end{proof}
    
\end{homeworkProblem}

\begin{homeworkProblem}
    \begin{enumerate}
        \item \(f^{-1}(\bigcup\limits_{\alpha \in I}Y_{\alpha}) = \left\{ x \in X: f(x) \in \bigcup\limits_{\alpha \in I}Y_{\alpha} \right\} = \left\{ x \in X: (\exists \alpha \in I: f(x) \in Y_{\alpha}) \right\}\)
        
        \(\bigcup\limits_{\alpha \in I}f^{-1}(Y_{\alpha}) = \left\{ x \in X: (\exists \alpha \in I: f(x) \in Y_{\alpha}) \right\}\)

        Since the two have identical expressions, \( f^{-1}(\bigcup\limits_{\alpha \in I}Y_{\alpha}) = \bigcup\limits_{\alpha \in I}f^{-1}(Y_{\alpha}) \)
        \item \(f^{-1}(\bigcap\limits_{\alpha \in I}Y_{\alpha}) = \left\{ x \in X: f(x) \in \bigcap\limits_{\alpha \in I}Y_{\alpha} \right\} = \left\{ x \in X: (\forall \alpha \in I: f(x) \in Y_{\alpha}) \right\}\)
        
        \(\bigcap\limits_{\alpha \in I}f^{-1}(Y_{\alpha}) = \left\{ x \in X: (\forall \alpha \in I: f(x) \in Y_{\alpha}) \right\}\)

        Since the two have identical expressions, \( f^{-1}(\bigcap\limits_{\alpha \in I}Y_{\alpha}) = \bigcap\limits_{\alpha \in I}f^{-1}(Y_{\alpha}) \)
    \end{enumerate}
\end{homeworkProblem}

\pagebreak

%
% Non sequential homework problems
%

% % Jump to problem 18
% \begin{homeworkProblem}[18]
%     Evaluate \(\sum_{k=1}^{5} k^2\) and \(\sum_{k=1}^{5} (k - 1)^2\).
% \end{homeworkProblem}

% % Continue counting to 19
% \begin{homeworkProblem}
%     Find the derivative of \(f(x) = x^4 + 3x^2 - 2\)
% \end{homeworkProblem}

% % Go back to where we left off
% \begin{homeworkProblem}[6]
%     Evaluate the integrals
%     \(\int_0^1 (1 - x^2) \dx\)
%     and
%     \(\int_1^{\infty} \frac{1}{x^2} \dx\).
% \end{homeworkProblem}



\end{document}