\documentclass[fleqn]{article}

\usepackage{fancyhdr}
\usepackage{extramarks}
\usepackage{amsmath, amssymb}

\usepackage{amsthm}
\usepackage{amsfonts}
\usepackage{tikz}
\usepackage[plain]{algorithm}
\usepackage{algpseudocode}

\setlength{\mathindent}{1cm}

\newcommand\N{\ensuremath{\mathbb{N}}}
\newcommand\R{\ensuremath{\mathbb{R}}}
\newcommand\Z{\ensuremath{\mathbb{Z}}}
\renewcommand\O{\ensuremath{\emptyset}}
\newcommand\Q{\ensuremath{\mathbb{Q}}}
\newcommand\C{\ensuremath{\mathbb{C}}}
\newcommand\Ha{\ensuremath{\mathbb{H}}}
\newcommand\J{\ensuremath{\mathbb{J}}}
\newcommand\K{\ensuremath{\mathbb{K}}}
\newcommand\cont{\Rightarrow\!\Leftarrow}
\newcommand\tf{\therefore}

\usetikzlibrary{automata,positioning}

%
% Basic Document Settings
%

\topmargin=-0.45in
\evensidemargin=0in
\oddsidemargin=0in
\textwidth=6.5in
\textheight=9.0in
\headsep=0.25in

\linespread{1.1}

\pagestyle{fancy}
\lhead{\hmwkAuthorName}
\chead{\hmwkClass\ : \hmwkTitle}
\rhead{\firstxmark}
\lfoot{\lastxmark}
\cfoot{\thepage}

\renewcommand\headrulewidth{0.4pt}
\renewcommand\footrulewidth{0.4pt}

\setlength\parindent{0pt}

%
% Create Problem Sections
%

\newcommand{\enterProblemHeader}[1]{
    \nobreak\extramarks{}{Problem \arabic{#1} continued on next page\ldots}\nobreak{}
    \nobreak\extramarks{Problem \arabic{#1} (continued)}{Problem \arabic{#1} continued on next page\ldots}\nobreak{}
}

\newcommand{\exitProblemHeader}[1]{
    \nobreak\extramarks{Problem \arabic{#1} (continued)}{Problem \arabic{#1} continued on next page\ldots}\nobreak{}
    \stepcounter{#1}
    \nobreak\extramarks{Problem \arabic{#1}}{}\nobreak{}
}

\setcounter{secnumdepth}{0}
\newcounter{partCounter}
\newcounter{homeworkProblemCounter}
\setcounter{homeworkProblemCounter}{1}
\nobreak\extramarks{Problem \arabic{homeworkProblemCounter}}{}\nobreak{}

%
% Homework Problem Environment
%
% This environment takes an optional argument. When given, it will adjust the
% problem counter. This is useful for when the problems given for your
% assignment aren't sequential. See the last 3 problems of this template for an
% example.
%
\newenvironment{homeworkProblem}[1][-1]{
    \ifnum#1>0
        \setcounter{homeworkProblemCounter}{#1}
    \fi
    \section{Problem \arabic{homeworkProblemCounter}}
    \setcounter{partCounter}{1}
    \enterProblemHeader{homeworkProblemCounter}
}{
    \exitProblemHeader{homeworkProblemCounter}
}

%
% Homework Details
%   - Title
%   - Due date
%   - Class
%   - Section/Time
%   - Instructor
%   - Author
%

\newcommand{\hmwkTitle}{Homework\ \#3}
\newcommand{\hmwkDueDate}{February 1, 2022}
\newcommand{\hmwkClass}{Math 131AH}
\newcommand{\hmwkClassTime}{131AH}
\newcommand{\hmwkClassInstructor}{Professor Marek Biskup}
\newcommand{\hmwkAuthorName}{\textbf{Nakul Khambhati}}

%
% Title Page
%

\title{
    \vspace{2in}
    \textmd{\textbf{\hmwkClass:\ \hmwkTitle}}\\
    \normalsize\vspace{0.1in}\small{Due\ on\ \hmwkDueDate}\\
    \vspace{0.1in}\large{\textit{\hmwkClassInstructor}}
    \vspace{3in}
}

\author{\hmwkAuthorName}
\date{}

\renewcommand{\part}[1]{\textbf{\large Part \Alph{partCounter}}\stepcounter{partCounter}\\}

%
% Various Helper Commands
%

% Useful for algorithms
\newcommand{\alg}[1]{\textsc{\bfseries \footnotesize #1}}

% For derivatives
\newcommand{\deriv}[1]{\frac{\mathrm{d}}{\mathrm{d}x} (#1)}

% For partial derivatives
\newcommand{\pderiv}[2]{\frac{\partial}{\partial #1} (#2)}

% Integral dx
\newcommand{\dx}{\mathrm{d}x}

% Alias for the Solution section header
\newcommand{\solution}{\textbf{\large Solution}}

% Probability commands: Expectation, Variance, Covariance, Bias
\newcommand{\E}{\mathrm{E}}
\newcommand{\Var}{\mathrm{Var}}
\newcommand{\Cov}{\mathrm{Cov}}
\newcommand{\Bias}{\mathrm{Bias}}

\begin{document}

\maketitle

\pagebreak

\begin{homeworkProblem}
    Recall \(\forall m,n \in \N\) we define \(m^0 = 1\) and \(m^{S(n)} = mm^{n}\).
    \begin{enumerate}
        \item Fix \(s \in \N\). We will prove the statement by induction on \(r\).
        \(P_0: m^{0+s} = m^{s} = 1m^{s} = m^0m^s\). Assume \(P_r\). Let's prove \(P_{r+1}\):
        \(m^{S(r)+s} = m^{S(r+s)} = mm^{r+s}=mm^rm^s = m^{S(r)}m^s\). By induction, the 
        statement follows for all \(r \in \N\).
        \item Now, fix \(r\) and proceed by induction on \(s\). \(P_0: m^{r0} = m^{0} = 1 = (m^{r})^0\). 
        Assume \(P_s\) and prove \(P_{s+1}: m^{rS(s)} = m^{rs + r} = m^{rs}m^r = (m^r)^sm^r = (m^{r})^{S(s)}\). 
        By induction, the result follows for all \(s \in \N\).
    \end{enumerate}
\end{homeworkProblem}

\begin{homeworkProblem}
    We will prove this by proposing some fact about \(A\) which we will prove for all \(n \in \N\).
    Let \(P_n\) be the statement: If \(A\) contains an integer \(k\) where \(0 \leq k \leq n\) then, \(A\) has a minimal element. 
    \(P_0\) is clear as if \(0 \in A\) then \(0\) is the minimal element. Now assume \(P_n\). We show \(P_{n+1}\). Assume \(A\) contains
    some \(k\) such that \(0 \leq k \leq n+1\). We now deal with cases: If there does not exist any \(a \in A \text{ such that } 0 \leq a \leq n\) then we must
    have \(k = n+1\) which would then be the minimal element in \(A\). On the other hand, if \(\exists a: 0 \leq a \leq n \) then by the inductive hypothesis, \(A\)
    has a minimal element. By induction we have proved \(P_n\) for all \(n \in \N\). Since we are given that \(A\) is nonempty, we can say that \(A\) contains some
    integer \(k: 0\leq k \leq m\) for some \(m \in \N\). As a result, \(A\) has a minimal element.
\end{homeworkProblem}

\begin{homeworkProblem}
    We will use slightly different notation to prove that this relation is well defined. 
    Assume that \([(m,n)]\sim [(m',n')]\) and \([(k,l)]\sim [(k', l')]\). 
    Further assume that \([(m,n)]\leq [(k,l)]\). We are required to show that \([(m',n')]\leq [(k', l')]\).\\

    First let's summarize our assumptions: \(m + n' = m' + n, k + l' = k' +l\) and \(m+l \leq n+k\). We need to show
    \(m' + l' \leq n' + k'\). In other word, there exists some \(a \in \N\) such that \(n + k = m + l + a\). We need to
    construct some \(a' \in \N: n' + k' = m' + l' + a'\). By adding some of the above equations, we get \(n+k+m+n'+k'+l = m+l+a+m'+n + k + l'\).
    Injectivity of addition allows us to cancel \(k,l,m,n\) from both sides leaving us with \(n' + k' = m' + l' + a\). So in fact, the required \(a'\)
    is \(a\) and this makes the relation well defined. \\

    First we show that the relation is a partial order:
    \begin{enumerate}
        \item By commutativity, \(m + n = n + m = n + m + 0\) so \([(m,n)]\leq [(m,n)]\).
        \item Assume that \([(m,n)]\leq [(m',n')]\) and \([(m',n')]\leq [(m,n)]\). Then, \(m + n' = m' + n + k\) and \(m' + n = m' + n + l\) for some \(k,l \in \N\).
        So, \(m + n' = m + n' + k + l\). By injectivity of addition, \(k+l = 0\) and \(k = l = 0\). So, \(m+n' = n+m' \Rightarrow [(m,n)]\sim [(m',n')]\). Therefore, the 
        relation is antisymmetric.
        \item Assume that \([(m,n)]\leq [(m',n')]\) and \([(m',n')]\leq [(m'',n'')]\). Then, \(m + n' = m' + n + k\) and \(m' + n'' = m'' + n' + l\) for some \(k,l \in \N\).
        When we add the two, we get \(m + n' + m' + n'' = m' + n + k + m'' + n' + l\). By cancellation, we cancel out \(m',n'\) from both sides and by commutativity we rearrange
        to get \(m + n'' = m'' + n + k+l\). So \([(m,n)]\leq [(m'',n'')]\) and the relation is transitive. 
    \end{enumerate}

    Now we need to show that for any two \([(m,n)], [(k,l)]\) either \([(m,n)] \leq [(k,l)]\) or \([(k,l)] \leq [(m,n)]\).\\
    We will prove this by using the total ordering of naturals. Given \([(m,n)]\) and \([(k,l)]\), either \(m+l \leq n+k \Rightarrow [(m,n)] \leq [(k,l)]\)
    or \(n+k \leq m+l \stackrel{comm}{\Rightarrow} k+n \leq l+m \Rightarrow [(k,l)] \leq [(m,n)]\). 

\end{homeworkProblem}

\begin{homeworkProblem}
    Again, we switch notation a bit to show that addition is well-defined. Let \([(p,q)]\sim [(p',q')]\) and \([(r,s)]\sim [(r',s')]\). Then, \(pq' = p'q\) and \(rs' = r's\). 
    We need to then show that \((ps +qr, qs)\sim(p's' + q'r', q's')\). LHS = \((ps + qr)(q's') = (psq's' + qrq's') = (p'qss' + r'sqq') = (p's' + q'r')(qs)\) = RHS. 
\end{homeworkProblem}

\begin{homeworkProblem}
    \begin{enumerate}
        \item Let \(a,b \in F\). Let \(0 \leq b\). Then by (O2) \(0 + a = a \leq a + b\). Conversely, 
        assume \(a  \leq a+b \Rightarrow a - a = 0 \leq a + b - a = b\).
        \item Let \(a,b \in F\). Then, \(a \leq b \Rightarrow a - a - b \leq b - a - b \Rightarrow -b \leq -a\).
        \item Let \(a,b \in F\). Assume \(0 < a \land a\leq b\). Therefore, \(a\neq 0 \land b \neq 0\). Therefore
        \(a^{-1}, b^{-1}\) exist. Further, \( 0 < a^{-1}, 0 < b^{-1}\) because otherwise we would get \(1 < 0\). Therefore, 
        \(aa^{-1}b^{-1} \leq ba^{-1}b^{-1} \Rightarrow b^{-1} \leq a^{-1}\).
    \end{enumerate}
\end{homeworkProblem}

\begin{homeworkProblem}
    These results follow from the proposition we proved in class: \(0 \leq a \iff -a \leq 0\).
    \begin{enumerate}
        \item If \(0 \leq a\), then \(|a| = a\) so \(0 \leq |a|\). Else, \(a < 0, |a| = -a, 0\leq |a|\).
        \item Assume \(0 \leq a\). Then \(-a \leq a \leq a\) so \(-|a| \leq a \leq |a|\). Else, \(a < 0, |a| = -a\). Then, 
        \(a \leq a \leq -a\) so \(-|a| \leq a \leq |a|\).
        \item If both \(a,b \geq 0\) then \(|a+b| = |a| + |b|\). Similarly, if both \(a,b < 0\) then \(|a+b| = -(a+b) = (-a) + (-b) = |a| + |b|\).
        If one is negative and one is positive: Assume WLOG \(0 \leq a\) and \(b < 0\). Then, \(|a+b| = |a - (-b)| \leq |a| + |b|\).
        \item Assume both \(a,b\) are positive. Then \(ab\) is positive so \(|ab| = ab = |a||b|\). Similarly, for both negative, \(ab\) is positive 
        so \(|ab| = ab = (-|a|)(-|b|) = |a||b|\). If only one is positive, assume WLOG \(0 \leq a\) and \(b < 0\). Then \(ab\) is negative so \(|ab| = -(ab) = -(|a|(-|b|)) = |a||b|\).
    \end{enumerate}

    Let's prove the claim by induction on \(n\). \(P_0\) is trivially true. Assume \(P_n\). Then, \(\Big|\sum\limits_{i = 0}^{n+1} a_i \Big| = \Big|\sum\limits_{i = 0}^{n}a_i + a_{n+1}\Big| \leq 
    \Big|\sum\limits_{i = 0}^{n}a_i \Big| + | a_{n+1}| \leq \sum\limits_{i = 0}^{n}|a_i| + |a_{n+1}| = \sum\limits_{i = 0}^{n+1}|a_i|\). By induction, the claim is true for all \(n \in \N\).
\end{homeworkProblem}

\begin{homeworkProblem}
    \begin{enumerate}
        \item The radical expression \(\sqrt[3]{5 - \sqrt{3}}\) solves some polynomial equation. Let's use \(\alpha\) to denote the radical expression. Then, \(5 - \sqrt{3} = \alpha^3 \Rightarrow
        (5-\alpha^3)^2 = 3\). So \(\alpha\) is a root of the polynomial \(p(x) = x^6 - 10x^3 + 22\). By the rational root theorem, the only possible rational roots
        of \(p(x)\) are \(\pm 1, \pm 2, \pm 11, \pm 22\). Substituting each, we see that none of them make the polynomial equal zero. Therefore \(\alpha\) cannot be expressed as a rational number. 
        \item We can rewrite this radical as \(\sqrt{3 + 2\sqrt{2}} - \sqrt{2} = \sqrt{(1+\sqrt{2})^2} - \sqrt{2} = 1 + \sqrt{2} - \sqrt{2} = 1\) which is rational.
        \item By the rational root theorem, the only possible roots for this polynomial are \(\pm 1\). We check that \(p(1) = 4, \; p(-1) = 0\). Therefore, \(-1\) is the only rational root of the
        polynomial. 
    \end{enumerate}
\end{homeworkProblem}

\end{document}