\documentclass[fleqn]{article}

\usepackage{fancyhdr}
\usepackage{extramarks}
\usepackage{amsmath, amssymb}

\usepackage{amsthm}
\usepackage{amsfonts}
\usepackage{tikz}
\usepackage[plain]{algorithm}
\usepackage{algpseudocode}

\setlength{\mathindent}{1cm}

\newcommand\N{\ensuremath{\mathbb{N}}}
\newcommand\R{\ensuremath{\mathbb{R}}}
\newcommand\Z{\ensuremath{\mathbb{Z}}}
\renewcommand\O{\ensuremath{\emptyset}}
\newcommand\Q{\ensuremath{\mathbb{Q}}}
\newcommand\C{\ensuremath{\mathbb{C}}}
\newcommand\Ha{\ensuremath{\mathbb{H}}}
\newcommand\J{\ensuremath{\mathbb{J}}}
\newcommand\K{\ensuremath{\mathbb{K}}}
\newcommand\cont{\Rightarrow\!\Leftarrow}
\newcommand\tf{\therefore}

\usetikzlibrary{automata,positioning}

%
% Basic Document Settings
%

\topmargin=-0.45in
\evensidemargin=0in
\oddsidemargin=0in
\textwidth=6.5in
\textheight=9.0in
\headsep=0.25in

\linespread{1.1}

\pagestyle{fancy}
\lhead{\hmwkAuthorName}
\chead{\hmwkClass\ : \hmwkTitle}
\rhead{\firstxmark}
\lfoot{\lastxmark}
\cfoot{\thepage}

\renewcommand\headrulewidth{0.4pt}
\renewcommand\footrulewidth{0.4pt}

\setlength\parindent{0pt}

%
% Create Problem Sections
%

\newcommand{\enterProblemHeader}[1]{
    \nobreak\extramarks{}{Problem \arabic{#1} continued on next page\ldots}\nobreak{}
    \nobreak\extramarks{Problem \arabic{#1} (continued)}{Problem \arabic{#1} continued on next page\ldots}\nobreak{}
}

\newcommand{\exitProblemHeader}[1]{
    \nobreak\extramarks{Problem \arabic{#1} (continued)}{Problem \arabic{#1} continued on next page\ldots}\nobreak{}
    \stepcounter{#1}
    \nobreak\extramarks{Problem \arabic{#1}}{}\nobreak{}
}

\setcounter{secnumdepth}{0}
\newcounter{partCounter}
\newcounter{homeworkProblemCounter}
\setcounter{homeworkProblemCounter}{1}
\nobreak\extramarks{Problem \arabic{homeworkProblemCounter}}{}\nobreak{}

%
% Homework Problem Environment
%
% This environment takes an optional argument. When given, it will adjust the
% problem counter. This is useful for when the problems given for your
% assignment aren't sequential. See the last 3 problems of this template for an
% example.
%
\newenvironment{homeworkProblem}[1][-1]{
    \ifnum#1>0
        \setcounter{homeworkProblemCounter}{#1}
    \fi
    \section{Problem \arabic{homeworkProblemCounter}}
    \setcounter{partCounter}{1}
    \enterProblemHeader{homeworkProblemCounter}
}{
    \exitProblemHeader{homeworkProblemCounter}
}

%
% Homework Details
%   - Title
%   - Due date
%   - Class
%   - Section/Time
%   - Instructor
%   - Author
%

\newcommand{\hmwkTitle}{Homework\ \#2}
\newcommand{\hmwkDueDate}{January 25, 2022}
\newcommand{\hmwkClass}{Math 131AH}
\newcommand{\hmwkClassTime}{131AH}
\newcommand{\hmwkClassInstructor}{Professor Marek Biskup}
\newcommand{\hmwkAuthorName}{\textbf{Nakul Khambhati}}

%
% Title Page
%

\title{
    \vspace{2in}
    \textmd{\textbf{\hmwkClass:\ \hmwkTitle}}\\
    \normalsize\vspace{0.1in}\small{Due\ on\ \hmwkDueDate}\\
    \vspace{0.1in}\large{\textit{\hmwkClassInstructor}}
    \vspace{3in}
}

\author{\hmwkAuthorName}
\date{}

\renewcommand{\part}[1]{\textbf{\large Part \Alph{partCounter}}\stepcounter{partCounter}\\}

%
% Various Helper Commands
%

% Useful for algorithms
\newcommand{\alg}[1]{\textsc{\bfseries \footnotesize #1}}

% For derivatives
\newcommand{\deriv}[1]{\frac{\mathrm{d}}{\mathrm{d}x} (#1)}

% For partial derivatives
\newcommand{\pderiv}[2]{\frac{\partial}{\partial #1} (#2)}

% Integral dx
\newcommand{\dx}{\mathrm{d}x}

% Alias for the Solution section header
\newcommand{\solution}{\textbf{\large Solution}}

% Probability commands: Expectation, Variance, Covariance, Bias
\newcommand{\E}{\mathrm{E}}
\newcommand{\Var}{\mathrm{Var}}
\newcommand{\Cov}{\mathrm{Cov}}
\newcommand{\Bias}{\mathrm{Bias}}

\begin{document}

\maketitle

\pagebreak

\begin{homeworkProblem}
\begin{proof}
        We need to use induction to prove that \(\forall n \geq 1: \sum\limits_{k = 1}^{n}k^3  = \Big(\sum\limits_{k = 1}^{n}k\Big)^2 \)\\
        Let's call this predicate \(P(n)\). First we show \(P(1)\). For \(n =1, \texttt{LHS} = \texttt{RHS} = 1\).\\
        Assume \(P(n)\), check \(P(n+1)\). \(\texttt{LHS} = \sum\limits_{k=1}^{n+1}k^3 = \sum\limits_{k=1}^{n}k^3 + (n+1)^3\). By \(P(n)\), this equals \(\Big(\sum\limits_{k = 1}^{n}k\Big)^2 + (n+1)^3\).
        Now, we use the formula for the sum of the first \(n\) integers: \(\sum\limits_{k=1}^{n}k = \dfrac{n(n+1)}{2}\). This simplifies the \texttt{LHS} to 
        \(\dfrac{n^2(n+1)^2}{4} + (n+1)^3 = \dfrac{(n+1)^2(n^2 + 4n + 4)}{4} = \left( \dfrac{(n+1)(n+2)}{2} \right)^2 = \left( \sum\limits_{k=1}^{n+1}k \right)^2 = \texttt{RHS}\).
        Then, by the principle of mathematical induction \(P(n)\) is \texttt{TRUE} \(\forall n\geq1\)
\end{proof}
\end{homeworkProblem}

\begin{homeworkProblem}
    Using Problem 1, we can express \(\sum\limits_{k=1}^n k^3\) as a polynomial of degree \(4\). 
    The hope here is to express \(\sum\limits_{k=1}^n k^4\) as a polynomial of degree \(5\). 
    Before solving this question, let's see if we can find an alternate way to directly prove Problem 1. 
    Then, we can generalize this method. \\

    Consider the polynomial \((x+1)^4 = x^4 + 4x^3 + 6x^2 + 4x + 1\)
    \begin{align*}
        &\Rightarrow (x+1)^4 - x^4 = 4x^3 + 6x^2 + 4x + 1 \\
        &\Rightarrow \sum\limits_{x = 1}^{n}(x+1)^4 - x^4 = \sum\limits_{x = 1}^{n}(4x^3 + 6x^2 + 4x + 1)  \\
        &\Rightarrow (n+1)^4 - 1 = \sum\limits_{x = 1}^{n}(4x^3) + \sum\limits_{x = 1}^{n}(6x^2) + \sum\limits_{x = 1}^{n}(4x) + \sum\limits_{x = 1}^{n}(1)
    \end{align*}

    Here the last implication follows from the telescoping sum on the \texttt{RHS} and distribution of the sum on the \texttt{RHS}.
    Recall, we have closed formulae for the sum of consecutive squares and consecutive integers so we can rearrange the sum to get an expression for \(\sum\limits_{x=1}^n x^3\) as required. We will use the same method to deduce a polynomial expression for \(\sum\limits_{k =1}^n k^4\).\\

    Now consider the polynomial \((x+1)^5 = x^5 + 5x^4 + 10x^3 + 10x^2 + 5x + 1\)
    \begin{align*}
        &\Rightarrow (x+1)^5 - x^5 = 5x^4 + 10x^3 + 10x^2 + 5x + 1 \\
        &\Rightarrow \sum\limits_{x = 1}^{n}(x+1)^5 - x^5 = \sum\limits_{x = 1}^{n}(5x^4 + 10x^3 + 10x^2 + 5x + 1)  \\
        &\Rightarrow (n+1)^5 - 1 = \sum\limits_{x = 1}^{n}(5x^4) + \sum\limits_{x = 1}^{n}(10x^3) + \sum\limits_{x = 1}^{n}(10x^2) + \sum\limits_{x = 1}^{n}(5x) + \sum\limits_{x = 1}^{n}(1) \\
        &\Rightarrow (n+1)^5 - 1 = 5\sum\limits_{x = 1}^{n}x^4 + 10\left( \dfrac{n^2(n+1)^2}{4} \right) + 10\left( \dfrac{n(n+1)(2n+1)}{6} \right) + 5\left( \dfrac{n(n+1)}{2} \right) + n \\
        &\Rightarrow \sum\limits_{x=1}^n x^4 = \dfrac{n(n+1)(2n+1)(3n^2+3n -1)}{30}
    \end{align*}

    Again, we can prove this using induction. 
    
    \begin{proof}
        Let \(P(n)\) denote the predicate we wish to prove. 
        \(P(1): \texttt{LHS} = 1, \texttt{RHS} = \dfrac{1\times 2\times 3\times 5}{30} = 1\). 
        Assume \(P(n)\), check \(P(n+1)\). As before, the \texttt{LHS} reduces to \(\dfrac{n(n+1)(2n+1)(3n^2+3n -1)}{30} + (n+1)^4 \\ = 
        (n+1)\left( \dfrac{n(2n+1)(3n^2+3n -1)+ 30(n+1)^3}{30} \right) = \dfrac{(n+1)(n+2)(2n+3)(3n^2 + 9n + 5)}{30} = \texttt{RHS}\)
    \end{proof}
    Then, by the principle of mathematical induction \(P(n)\) is \texttt{TRUE} \(\forall n\geq1\)
    

\end{homeworkProblem}

\begin{homeworkProblem}
    We are given \(P(n) : \sum\limits_{k=1}^n k  = \dfrac{1}{2}
    (n+\frac{1}{2})^2\). Let's assume \(P(n)\) and try to prove \(P(n+1)\). 
    \(\verb|LHS| = \sum\limits_{k=1}^n k + (n+1) = \dfrac{1}{2}(n+\frac{1}{2})^2 + (n+1)
    = \dfrac{1}{2}(n+\frac{3}{2})^2 = \verb|RHS|\). Therefore, \(P(n) \Rightarrow P(n+1)\). 
    However, this does not imply that \(P(n)\) is \verb|TRUE| for all \(n \in \N\) as the base case
    \(P_0\) is clearly false. In fact, one can see that \(P(n)\) is \verb|FALSE| for all \(n \in \N\) as the \verb|LHS|
    is always an integer while the \verb|RHS| always has a fractional component.
\end{homeworkProblem}

\begin{homeworkProblem}
    We define \(\J:=\left\{ n + \dfrac{m}{m+1} : n, m \in \N \right\}\) and \(S:\J \to \J\) as \(S\left( n+ \dfrac{m}{m+1} \right)
    = n + \dfrac{m+1}{m+2} \).\\
    Let us first verify that \((\J, 0, S)\) satsify (P1)-(P4).
    \begin{enumerate}
        \item Let \(n = 0, m = 0\). Therefore, \(0 \in \J\).
        \item \(S\) is clearly defined over all of \(\J\).
        \item \(n + \dfrac{m+1}{m+2} \geq \dfrac{m+1}{m+2} \geq \dfrac{1}{2}\) so \(0 \notin Ran(S)\).
        \item Assume \(a + \dfrac{b+1}{b+2} = c + \dfrac{d+1}{d+2}\). Denote \verb|LHS| as \(\alpha\) and \verb|RHS| as \(\beta\).
        We have \(a < \alpha < a+1\), \(c < \beta < c+1\) and \(\alpha = \beta\) with \(a,c \in \N\). Every rational has a unique 
        representation as a natural number and a proper fraction. Therefore, \(a = c\) and \(a+1 = c+1\). As a result \(a + \dfrac{b}{b+1} = c + \dfrac{d}{d+1}\) 
        so \(S\) is injective.
    \end{enumerate}
    We now show that (P5) fails by giving a subset \(A\subset \J: 0 \in A \land S(A)\in A \land (A \neq \J)\).
    Let \(A = Ran(S)\cup\left\{ 0 \right\}\). Then, by construction, \(0 \in A\). Since \(S(\J)\subset A\), it is
    clear that \(S(A)\subset A\). However, \(1 \in \J\) but \(1 \notin A\) so \(A \neq \J\).\\

    Now, consider \(\K \subset \J, \K = \left\{ \dfrac{m}{m+1} : m \in \N \right\}\). In other words, \(\K\) is the 
    subset obtained when \(n\) is fixed at \(0\). \(S\) is defined like above \(S\left( \dfrac{m}{m+1}\right) =  \dfrac{m+1}{m+2} \).
    Again, (P1) holds as \(0 \in \K\) by setting \(m = 0\). (P2) to (P4) hold as they hold in \(\J\) and the properties are preserved under taking subsets.
    Finally, (P5) holds here as it holds for \(\N\). 
\end{homeworkProblem}

\begin{homeworkProblem}
    Recall that we have defined \(m+n\) for \(m,n \in \N\) recursively as \(m+0 = m\) and \(\forall n \in \N: m+S(n) = S(m+n)\).
    We are asked to show that \(\forall k,m,n \in \N: n + (m+k) = (n+m)+k\). We will prove this via induction on \(k\). Fix \(n, m \in \N\). Then \(P(0): n + (m+0) = n + m = (n+m) + 0 \) 
    by the definition of addition with \(0\). Assume \(P(k)\). We need to show \(P(S(k)): n + (m + S(k)) = (n+m) + S(k)\). We start from the \verb|LHS|: \(n+(m+S(k)) \stackrel{def}{=} n + S(m+k) \stackrel{def}{=} S(n+(m+k))
    \stackrel{P(k)}{=} S((n+m)+k) \stackrel{def}{=} (n+m) + S(k)\). Therefore \(P(k) \Rightarrow P(S(k))\). By (P5) this is true for all \(k \in \N\).

\end{homeworkProblem}

\begin{homeworkProblem}
    We can solve this by constantly dividing by powers of \(6\) and noting the quotients.
    \begin{align*}
        10000_{10} &= 7776 + 1296 + 4\cdot 216 + 36 + 4\cdot 6 + 4 \\
            &= 6^5 + 6^4 + 4\cdot 6^3 + 6^2 + 4\cdot 6 + 4\\
            &= 114144_{6}
    \end{align*}
    Therefore, \(10000_{10} = 114144_{6}\).
\end{homeworkProblem}

\begin{homeworkProblem}
    We define the ordering relation \(\leq\) as \(m \leq n \iff \exists k\in \N : n = m+k\).
    \begin{enumerate}
        \item Let \(n \in \N\). We can always write \(n = 0 + n\) so \(0 \leq n\).
        \item Let \(n \in \N\). Consider \(S(n) = S(n + 0) \stackrel{def}{=} n + S(0)\). Therefore, \(n \leq S(n)\).
        \item Let \(m,n \in \N\) and assume \(m \leq n\). Then, \(\exists k\in \N\) such that \(n = m + k\). Then, \(S(n) = S(m+k) \stackrel{comm}{=} S(k+m) \stackrel{def}{=} k + S(m) \stackrel{comm}{=} S(m) + k\). 
        Therefore \(S(m) \leq S(n)\).
    \end{enumerate}

    Now, we recursively define \(n\cdot m\) as \(0\cdot m = 0\) and \(\forall n \in \N: S(n)\cdot m = n\cdot m + m\).\\
    We are required to prove that \(\forall m,n,r \in \N: m \leq n \Rightarrow r\cdot m \leq r\cdot n\). Assume that \(m \leq n\) so that 
    \(\exists k \in \N : n = m + k\). Let \(r \in \N\). Then, by distributivity, \(r\cdot n = r\cdot (m+k) = r\cdot m + r \cdot k\). Therefore, \(r\cdot m \leq r\cdot n\). 
\end{homeworkProblem}

\begin{homeworkProblem}
    Denote \([0,n):= \left\{ k \in \N:k < n \right\}\). Let \(m,n \in \N\), \(h : [0,n) \to [0,m)\).
    \begin{enumerate}
        \item Fix \(n \in \N\). We will prove the result by induction on \(m\). \(P(0): h:[0,n) \to [0,0)\) is injective \(\Rightarrow\) \(n \leq m\).
        Since we cannot define such a function, the statement is vaccuously true. Assume \(P(m)\). Let \(h: [0,n) \to [0,m+1)\) be an injection. 
        If \(h([0,n))\subset [0,m)\) then we can redefine \(h\) as an injection \([0,n)\to[0,m)\). Then, \(n \leq m \leq m+1\) and we are done. 
        Otherwise, \(\exists\) exactly one \(j \in [0,n): h(j) = m\). We can then reconstruct the map and create a new function \(\tilde{h}:[0,n-1)\to [0,m)\) which 
        is still an injection. Here we define \(n-1\) as \(S(n-1) = n\). Then, \(n-1 \leq m\) so \(n \leq m+1\). Therefore, by (P5), the statement is true for all \(m \in \N\).
        \item Fix \(n \in \N\). Once again, we induct on \(m\). \(P(0)\) is vaccuously true like before. Assume \(P(m)\). Consider \(h : [0,n)\to [0,m+1)\) a surjection. Then, 
        \(\exists k \in [0,n)\) such that \(h(k) = m\). We can remove \(k\) from a domain and reconstruct the function \(\tilde{h}\) in a way such that \(\tilde{h}: [0,n-1) \to [0, m)\)
        is surjective. Then, by the inductive hypothesis, \(m \leq n-1\) so \(S(m) \leq n\). By (P5), this holds for all \(m \in \N\).
        \item This follows for the previous parts. If \(h\) is bijective, then it is injective (so \(n \leq m\)) and surjective (so \(m \leq n\)).
        Then \(\exists k: m = n + k, \exists l: n = m + l\). By associativity, \(n = n + (k+l)\). Therefore, \(k + l = 0 \Rightarrow k = 0, l =0\). So, \(m = n\).
        
    \end{enumerate}
\end{homeworkProblem}

\end{document}