\documentclass[fleqn]{article}

\usepackage{fancyhdr}
\usepackage{extramarks}
\usepackage{amsmath, amssymb}

\usepackage{amsthm}
\usepackage{amsfonts}
\usepackage{tikz}
\usepackage[plain]{algorithm}
\usepackage{algpseudocode}

\setlength{\mathindent}{1cm}

\newcommand\N{\ensuremath{\mathbb{N}}}
\newcommand\R{\ensuremath{\mathbb{R}}}
\newcommand\Z{\ensuremath{\mathbb{Z}}}
\renewcommand\O{\ensuremath{\emptyset}}
\newcommand\Q{\ensuremath{\mathbb{Q}}}
\newcommand\C{\ensuremath{\mathbb{C}}}
\newcommand\Ha{\ensuremath{\mathbb{H}}}
\newcommand\J{\ensuremath{\mathbb{J}}}
\newcommand\K{\ensuremath{\mathbb{K}}}
\newcommand\cont{\Rightarrow\!\Leftarrow}
\newcommand\tf{\therefore}

\usetikzlibrary{automata,positioning}

%
% Basic Document Settings
%

\topmargin=-0.45in
\evensidemargin=0in
\oddsidemargin=0in
\textwidth=6.5in
\textheight=9.0in
\headsep=0.25in

\linespread{1.1}

\pagestyle{fancy}
\lhead{\hmwkAuthorName}
\chead{\hmwkClass\ : \hmwkTitle}
\rhead{\firstxmark}
\lfoot{\lastxmark}
\cfoot{\thepage}

\renewcommand\headrulewidth{0.4pt}
\renewcommand\footrulewidth{0.4pt}

\setlength\parindent{0pt}

%
% Create Problem Sections
%

\newcommand{\enterProblemHeader}[1]{
    \nobreak\extramarks{}{Problem \arabic{#1} continued on next page\ldots}\nobreak{}
    \nobreak\extramarks{Problem \arabic{#1} (continued)}{Problem \arabic{#1} continued on next page\ldots}\nobreak{}
}

\newcommand{\exitProblemHeader}[1]{
    \nobreak\extramarks{Problem \arabic{#1} (continued)}{Problem \arabic{#1} continued on next page\ldots}\nobreak{}
    \stepcounter{#1}
    \nobreak\extramarks{Problem \arabic{#1}}{}\nobreak{}
}

\setcounter{secnumdepth}{0}
\newcounter{partCounter}
\newcounter{homeworkProblemCounter}
\setcounter{homeworkProblemCounter}{1}
\nobreak\extramarks{Problem \arabic{homeworkProblemCounter}}{}\nobreak{}

%
% Homework Problem Environment
%
% This environment takes an optional argument. When given, it will adjust the
% problem counter. This is useful for when the problems given for your
% assignment aren't sequential. See the last 3 problems of this template for an
% example.
%
\newenvironment{homeworkProblem}[1][-1]{
    \ifnum#1>0
        \setcounter{homeworkProblemCounter}{#1}
    \fi
    \section{Problem \arabic{homeworkProblemCounter}}
    \setcounter{partCounter}{1}
    \enterProblemHeader{homeworkProblemCounter}
}{
    \exitProblemHeader{homeworkProblemCounter}
}

%
% Homework Details
%   - Title
%   - Due date
%   - Class
%   - Section/Time
%   - Instructor
%   - Author
%

\newcommand{\hmwkTitle}{Homework\ \#4}
\newcommand{\hmwkDueDate}{February 8, 2023}
\newcommand{\hmwkClass}{Math 131AH}
\newcommand{\hmwkClassTime}{131AH}
\newcommand{\hmwkClassInstructor}{Professor Marek Biskup}
\newcommand{\hmwkAuthorName}{\textbf{Nakul Khambhati}}

%
% Title Page
%

\title{
    \vspace{2in}
    \textmd{\textbf{\hmwkClass:\ \hmwkTitle}}\\
    \normalsize\vspace{0.1in}\small{Due\ on\ \hmwkDueDate}\\
    \vspace{0.1in}\large{\textit{\hmwkClassInstructor}}
    \vspace{3in}
}

\author{\hmwkAuthorName}
\date{}

\renewcommand{\part}[1]{\textbf{\large Part \Alph{partCounter}}\stepcounter{partCounter}\\}

%
% Various Helper Commands
%

% Useful for algorithms
\newcommand{\alg}[1]{\textsc{\bfseries \footnotesize #1}}

% For derivatives
\newcommand{\deriv}[1]{\frac{\mathrm{d}}{\mathrm{d}x} (#1)}

% For partial derivatives
\newcommand{\pderiv}[2]{\frac{\partial}{\partial #1} (#2)}

% Integral dx
\newcommand{\dx}{\mathrm{d}x}

% Alias for the Solution section header
\newcommand{\solution}{\textbf{\large Solution}}

% Probability commands: Expectation, Variance, Covariance, Bias
\newcommand{\E}{\mathrm{E}}
\newcommand{\Var}{\mathrm{Var}}
\newcommand{\Cov}{\mathrm{Cov}}
\newcommand{\Bias}{\mathrm{Bias}}

% New commands added by Nakul for 131AH
\newcommand{\ps}{\mathcal{P}}

\begin{document}

\maketitle

\pagebreak

\begin{homeworkProblem}

    Let \(F\) be a non-empty set, \(E = \ps(F)\). We have already seen that
    this is a poset via \(\subset\). We are asked to show that every \(A \subset \ps(F)\) i.e.
    every collection of subsets of \(F\) admits an infimum and supremum. First, we consider nonempty collections. \\

    Let \(A \neq \O\). We show that \(\sup(A) = \bigcup A\). By definition, \(\forall X \in A: X \subset \bigcup A\).
    Now, let \(B\) be another upper bound for \(A\). Then, \(\forall X \in A: X \subset B \Rightarrow \bigcup A \subset B\).
    This shows that \(\bigcup A\) is the least upper bound for \(A\) i.e. \(\bigcup A= \sup(A)\).\\

    Next, we show \(\inf(A) = \bigcap A\). By definition, \(\forall X \in A: \bigcap A \subset X\) so it is a lower bound.
    Let \(C\) be another lower bound for \(A\) i.e. \(\forall X \in A: C \subset A \Rightarrow C \subset \bigcap A\). So,
    \(\bigcap A\) is the greatest lower bound i.e. \(\inf(A) = \bigcap A\).\\

    Finally, we consider \(A = \O\). To prove that \(\sup(\O) = \O\) it suffices to show that \(\O\) is an upper bound. 
    It will definitely be the smallest one as it is the minimal element of this poset. Similarly, to show \(\inf(\O) = F\) it
    suffices to show that \(F\) is a lower bound. To show that \(\sup(\O) = \O\), we require: \(\forall X \in \O: X \subset \O\). Since \(\O\) is
    the emptyset, there exists no such \(X\) that can be compared. As a result, the statement is vacuously true. Similarly, \(\forall X \in \O: F \subset X\)
    is vacuously true so \(\inf(\O) = F\).

\end{homeworkProblem}


\begin{homeworkProblem}
    We define \(m|n: = (\exists k \in \N' : n = mk)\). We need to prove that this a partial order. 
    \begin{enumerate}
        \item \(\forall m: m = 1m\) which implies that \(m|m\)
        \item Let \(m|n\) and \(n|m\). Then, \(\exists k,l \in \N': m = kn, n = lm\). Therefore, \(n = lkn\). Since \(n \neq 0, 1 = lk\).
        In \(\N'\) this implies that \(l = k = 1\). Therefore, \(m = n\).
        \item Let \(m|n\) and \(n|k\). So, \(n = c_1m\), \(k = c_2n\) for some \(c_1, c_2 \in \N'\). Therefore, 
        \(k = c_1c_1m\) so \(m|k\).
    \end{enumerate} 
    Therefore this is a partial order. Now we show that \(\inf(A)\) exists and \(\sup(A)\) exists if \(A\) is bounded.

    Let \(A \subset \N'\) such that \(A \neq \O\). It is clear that this set has a lower bound of \(1\) since \(\forall m \in \N': m = 1m \) so \(1|m\). For each \(a \in A\) we can consider its prime factorization into some 
    form \(a = p_1^{k_1}\cdots p_n^{k_n}\). We can do this for all elements in \(A\) and then for each prime \(p_i\) listed, we look at the powers \(k_i\) that it corresponds to for various \(a \in A\). This gives us a set of
    naturals indexed by \(A \subset \N\). Therefore, we can consider the minimal element (ordered by \(\leq\)) of each such set, call it \(m_i\). If all primes listed are \(p_1, \ldots, p_s\) then \(\inf(A) = p_1^{m_1}\cdots p_s^{m_s}\). This is because,
    by construction, each \(p_i^{m_i}\) divides all \(a \in A\). Also, if some \(l\) divides all \(a \in A\). Then, it must divide each of the prime factors \(p_i^{m_i}\) listed earlier. In other words, \(l|p_i^{m_i}\) for each \(i\) so
    \(l | p_1^{m_1}\cdots p_s^{m_s}\). This concludes our proof as we have shown a greatest lower bound.
    \\

    
    Similarly, let \(A\subset \N'\) be a bounded subset. Then, we denote \(u = \prod\limits_{x\in A} x\). Since \(A\) is bounded, this is a finite product and we can
    write \(u = x_1x_2\cdots x_n\). This is an upper bound on \(A\) as \(\forall x \in A: x | u\). Therefore, we can now consider the set \(U\) of upper bounds (ordered by \(|\)) on \(A\) which is nonempty. This is a subset of \(\N\) and therefore
    has a minimal element when ordered by \(\leq\). Let's call this \(\min(U)\). We claim that \(\sup(A) = \min(U)\) when \(A\) is ordered by \(|\). We will prove this by contradiction. This is because \(\sup(A)\) must lie in \(U\). Therefore, 
    if it is not the minimal element of \(U\), there is an element that is a upper bound on \(A\), call it \(u_0\) such that \(u_0 \leq\) \(\sup(A)\). This would in turn imply that \(u_0|\sup(A)\) which would contradict its minimality. 

\end{homeworkProblem}

\begin{homeworkProblem}
    \(\Rightarrow\) Assume that \(A\) is Dedekind infinite. Let \(f: A \to A\) be an injective map such that \(\exists a \in A: a \notin Im(f)\). We can then construct a map \(i: \N \to A\)
    by recursive construction which is injective. This proves that \(A\) is unbounded as \(\N\) can be identified as a subset of \(A\). For the construction, we first set \(i(0) = a\). We then
    set \(i(n) = f(i(n-1))\). We just need to verify that this is injective. Fix \(n \in \N\). We will show inductively that \(\forall m \neq n\) or WLOG \(\forall m < n : i(m) \neq i(n)\). The base case
    is true since \(i(0) = x \notin Im(f), i(n) \in Im(f)\). For the inductive step, we use the fact that \(f\) is injective. Assume \(f(m+1) = f(n)\) for \(m + 1 < n\). But then \(f(i(m)) = f(i(n-1))\) so \(i(m) = i(n-1)\) which
    contradicts the the inductive hypothesis. This proves that \(i\) is injective and \(A\) is unbounded.\\
    
    \(\Leftarrow\) Assume that \(A\) is unbounded. We first show that we can construct \(f: \N \to A\) injective. Since \(A\) is unbounded, it is nonempty so choose \(a \in A\). Then, set \(f(0) = a\). We can 
    recursively construct the map by choosing for \(f(n)\) some element in \(A\) that hasn't been chosen before i.e. \(f(n) \notin \left\{ f(0), f(1), \ldots f(n-1) \right\}\). This is permitted by the axiom of 
    choice and since \(A\) is unbounded. This map, by construction, is injective. We can easily modify this map to get another map \(b: A \to A\) where \(b\) is injective but not surjective. We construct \(b\) by simply 
    shifting the values in the image of \(f\). Explicitly, \(b(f(0)) = f(1)\) and \(b(f(n)) = b(f(n+1))\). This map is injective since \(f\) is and is not surjective since \(f(0) \notin Im(b)\).

\end{homeworkProblem}

\begin{homeworkProblem}
    Let \(A, B \subset \Q\) such that \(\sup(A), \sup(B)\) exist and \(A \subset B\).
    Since \(\sup(B)\) exists, \(\forall b \in B: b \leq \sup(B)\). In particular, 
    \(\forall a \in A: a \leq \sup(B)\). Therefore, \(\sup(B)\) is an upper bound for \(A\) as well. 
    By definition of \(\sup(A)\), this must mean that \(\sup(A) \leq \sup(B)\)\\

    Under these conditions, we can consider \(\sup(A\cup B)\). We show that this exists. 
    We know that \(\sup(A)\) is an upper bound for \(A\) and \(\sup(B)\) is an upper bound
    for \(B\). Therefore, \(\forall x \in A\cup B: x \leq \sup(A) \lor x \leq \sup(B) \Rightarrow 
    x \leq \max\left\{ \sup(A), \sup(B) \right\}\). Clearly, \(\max\left\{ \sup(A), \sup(B) \right\}\) is an upper bound for \(A\cup B\)
    
    Now, let \(u\) be another upper bound on \(A\cup B\). Then, \(\forall x \in A\cup B: x \leq u\). Since \(A \subset A\cup B\),
    \(\sup(A) \leq u\). This follows from the previous part: if \(A \subset B\), then \(\sup(A) \leq \sup(B) \leq \text{any upper bound on B}\).
    Similarly, \(B \subset A\cup B \Rightarrow \sup(B) \leq u\). Therefore, \(\max\left\{ \sup(A), \sup(B) \right\} \leq u\). This proves that 
    \(\sup(A \cup B) = \max\left\{ \sup(A), \sup(B) \right\}\).

    % From the previous part,
    % since \(A \subset A\cup B\), \(\sup(A)\leq \)
\end{homeworkProblem}


\begin{homeworkProblem}
    Consider \(A + B:= \left\{ a + b, a \in A \land b \in B \right\}\). Assume that \(\sup(A)\) and \(\sup(B)\) exist. Let \(a \in A, b \in B\). 
    Then \(a \leq \sup(A), b \leq \sup(B)\) so \(a + b \leq \sup(A) + \sup(B)\). So, \(\sup(A) + \sup(B)\) is an upper bound on \(A+B\). Now, let
    \(u\) be another upper bound on \(A+B\). So, \(\forall a,b: a + b \leq u\). Then we can write \(u \geq l + k\) where \(l,k\) are upper bounds on \(A\) and \(B\) respectively. Then, \(l+k\) is an
    upper bound on \(A+B\). Therefore, \(\sup(A) + \sup(B) \leq l + k \leq u\).  By transitivity, \(\sup(A) + \sup(B) \leq u\) so \(\sup(A+B) = \sup(A) + \sup(B)\).

\end{homeworkProblem}

\begin{homeworkProblem}
    Let \(A\) have a lower bound \(l\) i.e. \(\forall a \in A: l \leq a\). 
    Now consider the set \(-A = \left\{ -a: a \in A \right\}\). Then, 
    \(\forall (-a) \in -A: -a \leq -l\). Therefore, \(-l\) is an upper bound on the set \(-A\). 
    By the completeness of \(\R\), \(\sup(-A)\) exists. This means that for any upper bound \(u\)
    of \(-A\), \(\sup(-A) \leq u\). Therefore, for any lower bound \(l\) of \(A\), 
    \( l \leq -\sup(-A)\). Therefore, \(\inf(A) = -\sup(-A)\).
\end{homeworkProblem}

\begin{homeworkProblem}
    We need to show that we cannot define an order on \(\C\). First note that \(0 \neq i\) Assume that in some ordering we have \(0 < i\). Then since \(i\)
    is positive, multiplication by \(i\) preserves the ordering relation: \(0 < i^2 \Rightarrow 0 < -1\) which is a contradiction. Instead,
    assume that \(i < 0\). Then \(i^2 > 0 \Rightarrow -1 > 0\) which is again a contradiction. Therefore, it is impossible to make \(C\) an ordered field. 
    As the hint suggests, this is because \(-1\) is a square and by our axioms of of an ordered field, a square must always be positive.
\end{homeworkProblem}

\end{document}