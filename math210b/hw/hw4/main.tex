\documentclass{article}

\usepackage{fancyhdr}
\usepackage{extramarks}
\usepackage{amsmath}
\usepackage{amsthm}
\usepackage{amsfonts}
\usepackage{amssymb}
\usepackage{tikz}
\usepackage[plain]{algorithm}
\usepackage{algpseudocode}
\usepackage{tikz-cd}

\newcommand\N{\ensuremath{\mathbb{N}}}
\newcommand\R{\ensuremath{\mathbb{R}}}
\newcommand\Z{\ensuremath{\mathbb{Z}}}
\renewcommand\O{\ensuremath{\emptyset}}
\newcommand\Q{\ensuremath{\mathbb{Q}}}
\newcommand\C{\ensuremath{\mathbb{C}}}
\newcommand\Ha{\ensuremath{\mathbb{H}}}
\newcommand\cont{\Rightarrow\!\Leftarrow}
\newcommand{\tp}[2]{#1 \otimes_R #2}

\usetikzlibrary{automata,positioning}

\newtheorem{theorem}{Theorem}[section]
\newtheorem{corollary}{Corollary}[theorem]
\newtheorem{lemma}[]{Lemma}

%
% Basic Document Settings
%

\topmargin=-0.45in
\evensidemargin=0in
\oddsidemargin=0in
\textwidth=6.5in
\textheight=9.0in
\headsep=0.25in

\linespread{1.1}

\pagestyle{fancy}
\lhead{\hmwkAuthorName}
\chead{\hmwkClass\ : \hmwkTitle}
\rhead{\firstxmark}
\lfoot{\lastxmark}
\cfoot{\thepage}

\renewcommand\headrulewidth{0.4pt}
\renewcommand\footrulewidth{0.4pt}

\setlength\parindent{0pt}

%
% Create Problem Sections
%

\newcommand{\enterProblemHeader}[1]{
    \nobreak\extramarks{}{Problem \arabic{#1} continued on next page\ldots}\nobreak{}
    \nobreak\extramarks{Problem \arabic{#1} (continued)}{Problem \arabic{#1} continued on next page\ldots}\nobreak{}
}

\newcommand{\exitProblemHeader}[1]{
    \nobreak\extramarks{Problem \arabic{#1} (continued)}{Problem \arabic{#1} continued on next page\ldots}\nobreak{}
    \stepcounter{#1}
    \nobreak\extramarks{Problem \arabic{#1}}{}\nobreak{}
}

\setcounter{secnumdepth}{0}
\newcounter{partCounter}
\newcounter{homeworkProblemCounter}
\setcounter{homeworkProblemCounter}{1}
\nobreak\extramarks{Problem \arabic{homeworkProblemCounter}}{}\nobreak{}

%
% Homework Problem Environment
%
% This environment takes an optional argument. When given, it will adjust the
% problem counter. This is useful for when the problems given for your
% assignment aren't sequential. See the last 3 problems of this template for an
% example.
%
\newenvironment{homeworkProblem}[1][-1]{
    \ifnum#1>0
        \setcounter{homeworkProblemCounter}{#1}
    \fi
    \section{Problem \arabic{homeworkProblemCounter}}
    \setcounter{partCounter}{1}
    \enterProblemHeader{homeworkProblemCounter}
}{
    \exitProblemHeader{homeworkProblemCounter}
}

%
% Homework Details
%   - Title
%   - Due date
%   - Class
%   - Section/Time
%   - Instructor
%   - Author
%

\newcommand{\hmwkTitle}{Homework\ \#4}
\newcommand{\hmwkDueDate}{February 9, 2023}
\newcommand{\hmwkClass}{Math 210B}
\newcommand{\hmwkClassTime}{210B}
\newcommand{\hmwkClassInstructor}{Professor Alexander Merkurjev}
\newcommand{\hmwkAuthorName}{\textbf{Nakul Khambhati}}

%
% Title Page
%

\title{
    \vspace{2in}
    \textmd{\textbf{\hmwkClass:\ \hmwkTitle}}\\
    \normalsize\vspace{0.1in}\small{Due\ on\ \hmwkDueDate}\\
    \vspace{0.1in}\large{\textit{\hmwkClassInstructor}}
    \vspace{3in}
}

\author{\hmwkAuthorName}
\date{}

\renewcommand{\part}[1]{\textbf{\large Part \Alph{partCounter}}\stepcounter{partCounter}\\}

%
% Various Helper Commands
%

% Useful for algorithms
\newcommand{\alg}[1]{\textsc{\bfseries \footnotesize #1}}

% For derivatives
\newcommand{\deriv}[1]{\frac{\mathrm{d}}{\mathrm{d}x} (#1)}

% For partial derivatives
\newcommand{\pderiv}[2]{\frac{\partial}{\partial #1} (#2)}

% Integral dx
\newcommand{\dx}{\mathrm{d}x}

% Alias for the Solution section header
\newcommand{\solution}{\textbf{\large Solution}}

% Probability commands: Expectation, Variance, Covariance, Bias
\newcommand{\E}{\mathrm{E}}
\newcommand{\Var}{\mathrm{Var}}
\newcommand{\Cov}{\mathrm{Cov}}
\newcommand{\Bias}{\mathrm{Bias}}

\begin{document}

\maketitle

\pagebreak

\begin{homeworkProblem}
    Let \(M\) be a left module that is generated by some \(a \in M\) i.e. \(M = Ra\). Consider the \(R\)-linear map 
    \(f: R \to M\) where \(1 \mapsto a\). (Recall that \(1\) generates \(R\) when \(R\) is viewed as a module over itself.)
    This map is surjective onto \(M\) since we can write each \(m = ra = f(r)\) for some \(r \in R\). Therefore, by the first isomorphism theorem, \(M \cong R/\ker(I)\).
    Since \(\ker(f) \subset R\) submodule, it is an ideal. Therefore, \(M \cong R/I\).
\end{homeworkProblem}

\begin{homeworkProblem}
    We define a relation on \(M\times S\) as \((m,s)\sim (m',s') \iff \exists u \in S: u(ms' - m's) = 0\). Addition and scalar multiplication are defined
    in the usual way. It is clear that this makes \(S^{-1}M\) an \(S^{-1}R\)-module. Let \(L: RMod \to S^{-1}RMod\) be the functor that takes an \(R\)-module
    to its localization (which is a module over the localized ring). We need to show this is functor is exact. 
    Consider the following exact sequence.  
    \begin{tikzcd}
        0 \arrow[r] &L \arrow[r, "u"] & M \arrow[r, "v"] & N \arrow[r] &0. 
    \end{tikzcd}
    Now consider the sequence after the localization functor is applied. 
    \begin{tikzcd}
        0 \arrow[r] &S^{-1}L \arrow[r, "u"] & S^{-1}M \arrow[r, "v"] & S^{-1}N \arrow[r] &0. 
    \end{tikzcd}
    Let \(x/s \in S^{-1}M: x/s \in \ker(v)\) i.e. \(\exists t\in S: v(xt) = v(x)t = 0 \in M\) i.e. \(xt \in \ker(v)\). By exactness, \(xt = u(y)\) for \(y \in L\).
    Then \(x/s\) is the image of \(y/st\) which proves exactness of this sequence as well. 
\end{homeworkProblem}

\begin{homeworkProblem}
    Let \(J \subset R\) and consider \(JM \subset M\). Then the quotient module
    \(M/JM\) is annihilated by \(J\). We define the \(R/J\) action on \(M/MJ\) as
    \(\overline{r} \overline{m} = rm\). This map is well-defined if and only if \(J\) is contained in the annihilator
    of \(M/JM\) which is the case here.  
\end{homeworkProblem}

\begin{homeworkProblem}
    Let \(F\) be a free module with basis \((f_i)_{i \in I}\) i.e. we can write every \(x \in F\) as \(x = \sum_I r_if_i\) for \(r_i \in R\). Let \(R \subset J\). From the previous problem, 
    we saw that we can give a natural structure to \(F/JF\) as an \(R/J\)-module. It follows that the set \((f_i + JF)_{i \in I}\) is a basis for \(F/JF\) since \(\sum_I \overline{r_i} \overline{f_i}
    = \sum_I r_if_i\).
\end{homeworkProblem}

\begin{homeworkProblem}
    We will solve this using the previous parts. Assume that \(R^n \cong R^m\). We know that \(\exists M \subset R\) a maximal ideal. 
    As a result, we can view \(R^n / MR^n \cong R^m / MR^m\) as \(R/M\)-modules i.e. vector spaces over a field. This reduces to \((R/MR)^m \cong (R/MR)^n\)
    as vector spaces so \(m = n\).
\end{homeworkProblem}


\begin{homeworkProblem}
    
    Assume \(A\) is a nonzero abelian group where \(A\oplus A \cong A\). Let \(R = End(A)\). We have an isomorphism \(\alpha: A^2 \to A\). We need to extend this to an 
    isomorphism \(\alpha': R^2 \to R\). Then the result follows inductively. 
    We construct the map as follows: \((f,g)(x) \mapsto \alpha(f,g)(x)\). This map has domain \(R^2\) and codomain \(R\). It is a ring homomorphism as 
    \(\alpha((f_1,g_1) + (f_2, g_2)) = \alpha(f_1 + f_2, g_1 + g_2) = \alpha(f_1, g_1) + \alpha(f_2, g_2)\). It follows that \(\alpha'\) is bijective since \(\alpha\) is bijective. 
    As a result, \(R^2 \cong R\) and by induction \(R^m \cong R \cong R^n\).

\end{homeworkProblem}

\begin{homeworkProblem}
    We prove that the arbitrary product of injective modules is injective. Recall that a module \(N\) is injective if every diagram of the form below can be extended as shown to a map from a larger module.
    \begin{tikzcd}
        0  \arrow[r] & A \arrow[r, hook] \arrow[d] & B \arrow[ld, dashed, "\exists"]\\
        & N
    \end{tikzcd}\\
    We are given a family of injective modules \((N_i)_{i\in I}\) such that the above property holds for all \(N_i\). Fix some \(A,B\). 

    Now consider the diagram \\
    \begin{tikzcd}
        0  \arrow[r] & A \arrow[r, hook] \arrow[d] & B \\
        & \prod N_i
    \end{tikzcd}\\

    By composing the canonical projection map \(p_i: \prod N_i \to N_i\), we get a diagram of the form below. \\
    \begin{tikzcd}
        0  \arrow[r] & A \arrow[r, hook] \arrow[d] & B \arrow[ld, dashed, "\exists \beta_i"]\\
        & N_i
    \end{tikzcd}\\
    Let's call the induced map \(\beta_i\). Finally, by universality, the maps \(\beta_i\) factor through \(\prod N_i\) via some map \(\beta: B \to \prod N_i\) which makes the previous diagram commute. 
    Therefore, \(\prod N_i\) is injective. By duality, this argument can be extended to coproducts of projective modules being projective (using the same universal properties).
\end{homeworkProblem}

\begin{homeworkProblem}
    A module \(P\) is projective if every short exact sequence with \(P\) at the end splits. A module \(N\) is injective if every short exact sequence with \(N\) at the front splits. 
    Therefore, every module is projective \(\iff\) every short exact sequence of modules splits \(\iff\) every module is injective. 
\end{homeworkProblem}

\begin{homeworkProblem}
    We will prove that \(P\) (the module described) is projective by showing that it is the direct summand of a free module. 
    To do this, we need to a construct an exact sequence that splits. A natural map that gives rise to our module is \(f: R^n \to R\) where \((x_1,\ldots, x_n)\mapsto a_1x_1 + \cdots + a_nx_n\)
    which is an \(R\)-linear map. The kernel of  the map is \(\ker(f) = P\). Further, this map is surjective as \((a_1, \ldots, a_n)\) generate \(R\). This gives us the exact sequence

    \begin{tikzcd}
        0 \arrow[r] & P \arrow[r] &R^n \arrow[r, "f"] &R \arrow[r] &0\\
    \end{tikzcd}\\
    which splits since \(R\) is free. Therefore, \(R^n = R\oplus P\) and \(P\) is projective.

\end{homeworkProblem}

\begin{homeworkProblem}
    In lecture, we saw that the functor \(\tp{-}{N}\) has right-adjoint \(Hom_{\Z}(N, -)\) since \(Hom(\tp{M}{N}, P) \cong Hom_R(M, Hom_{\Z}(N,P)) \) for all \(P \in AbGroups\). Since it has a right adjoint,
    this functor commutes with colimits (which we proved last quarter).
\end{homeworkProblem}

\end{document}