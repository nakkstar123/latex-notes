\documentclass{article}

\usepackage{fancyhdr}
\usepackage{extramarks}
\usepackage{amsmath}
\usepackage{amsthm}
\usepackage{amsfonts}
\usepackage{tikz}
\usepackage[plain]{algorithm}
\usepackage{algpseudocode}

\newcommand\N{\ensuremath{\mathbb{N}}}
\newcommand\R{\ensuremath{\mathbb{R}}}
\newcommand\Z{\ensuremath{\mathbb{Z}}}
\renewcommand\O{\ensuremath{\emptyset}}
\newcommand\Q{\ensuremath{\mathbb{Q}}}
\newcommand\C{\ensuremath{\mathbb{C}}}
\newcommand\Ha{\ensuremath{\mathbb{H}}}
\newcommand\cont{\Rightarrow\!\Leftarrow}

\usetikzlibrary{automata,positioning}

\newtheorem{theorem}{Theorem}[section]
\newtheorem{corollary}{Corollary}[theorem]
\newtheorem{lemma}[]{Lemma}

%
% Basic Document Settings
%

\topmargin=-0.45in
\evensidemargin=0in
\oddsidemargin=0in
\textwidth=6.5in
\textheight=9.0in
\headsep=0.25in

\linespread{1.1}

\pagestyle{fancy}
\lhead{\hmwkAuthorName}
\chead{\hmwkClass\ : \hmwkTitle}
\rhead{\firstxmark}
\lfoot{\lastxmark}
\cfoot{\thepage}

\renewcommand\headrulewidth{0.4pt}
\renewcommand\footrulewidth{0.4pt}

\setlength\parindent{0pt}

%
% Create Problem Sections
%

\newcommand{\enterProblemHeader}[1]{
    \nobreak\extramarks{}{Problem \arabic{#1} continued on next page\ldots}\nobreak{}
    \nobreak\extramarks{Problem \arabic{#1} (continued)}{Problem \arabic{#1} continued on next page\ldots}\nobreak{}
}

\newcommand{\exitProblemHeader}[1]{
    \nobreak\extramarks{Problem \arabic{#1} (continued)}{Problem \arabic{#1} continued on next page\ldots}\nobreak{}
    \stepcounter{#1}
    \nobreak\extramarks{Problem \arabic{#1}}{}\nobreak{}
}

\setcounter{secnumdepth}{0}
\newcounter{partCounter}
\newcounter{homeworkProblemCounter}
\setcounter{homeworkProblemCounter}{1}
\nobreak\extramarks{Problem \arabic{homeworkProblemCounter}}{}\nobreak{}

%
% Homework Problem Environment
%
% This environment takes an optional argument. When given, it will adjust the
% problem counter. This is useful for when the problems given for your
% assignment aren't sequential. See the last 3 problems of this template for an
% example.
%
\newenvironment{homeworkProblem}[1][-1]{
    \ifnum#1>0
        \setcounter{homeworkProblemCounter}{#1}
    \fi
    \section{Problem \arabic{homeworkProblemCounter}}
    \setcounter{partCounter}{1}
    \enterProblemHeader{homeworkProblemCounter}
}{
    \exitProblemHeader{homeworkProblemCounter}
}

%
% Homework Details
%   - Title
%   - Due date
%   - Class
%   - Section/Time
%   - Instructor
%   - Author
%

\newcommand{\hmwkTitle}{Homework\ \#1}
\newcommand{\hmwkDueDate}{January 18, 2022}
\newcommand{\hmwkClass}{Math 210B}
\newcommand{\hmwkClassTime}{210B}
\newcommand{\hmwkClassInstructor}{Professor Alexander Merkurjev}
\newcommand{\hmwkAuthorName}{\textbf{Nakul Khambhati}}

%
% Title Page
%

\title{
    \vspace{2in}
    \textmd{\textbf{\hmwkClass:\ \hmwkTitle}}\\
    \normalsize\vspace{0.1in}\small{Due\ on\ \hmwkDueDate}\\
    \vspace{0.1in}\large{\textit{\hmwkClassInstructor}}
    \vspace{3in}
}

\author{\hmwkAuthorName}
\date{}

\renewcommand{\part}[1]{\textbf{\large Part \Alph{partCounter}}\stepcounter{partCounter}\\}

%
% Various Helper Commands
%

% Useful for algorithms
\newcommand{\alg}[1]{\textsc{\bfseries \footnotesize #1}}

% For derivatives
\newcommand{\deriv}[1]{\frac{\mathrm{d}}{\mathrm{d}x} (#1)}

% For partial derivatives
\newcommand{\pderiv}[2]{\frac{\partial}{\partial #1} (#2)}

% Integral dx
\newcommand{\dx}{\mathrm{d}x}

% Alias for the Solution section header
\newcommand{\solution}{\textbf{\large Solution}}

% Probability commands: Expectation, Variance, Covariance, Bias
\newcommand{\E}{\mathrm{E}}
\newcommand{\Var}{\mathrm{Var}}
\newcommand{\Cov}{\mathrm{Cov}}
\newcommand{\Bias}{\mathrm{Bias}}

\begin{document}

\maketitle

\pagebreak

\begin{homeworkProblem}

    \begin{proof}
        Let $R$ be a finite integral domain i.e. $R\neq 0 $ and it has no zero divisors. In other words, if $xy = 0$, then $x=0$ or $y=0$.\\
    
        To prove that $R$ is a field, we need to show that $\forall r\in R, r\neq 0: \exists r^{-1} \in R$. 
        Let $r \in R$ be a non-zero element of $R$. Consider the map of sets $m_r: R \to R$ such that $s\mapsto rs$. We show that this map is injective. 
        Let $rs_1 = rs_2$. So $rs_1 - rs_2 = 0$ and $r(s_1 - s_2) = 0$. 
        Since we are working within a domain and we picked $r \neq 0$, we get that $s_1 - s_2 = 0$ so $s_1 = s_2$. \\
    
        We have proved that a map from $R$ to $R$ is injective. Since $R$ is finite, this map is also surjective. So, $\exists r' \in R: rr' = 1$ so $r^{-1} = r'$ and $r$ has an inverse. This shows that every non-zero element of $R$ has an inverse, and it is a field. 
    \end{proof}

\end{homeworkProblem}

\begin{homeworkProblem}
    
\begin{proof}
    Let \(A \in M_n(R)\). Then, \(det(A) = 0\) is equivalent to there being an expression (the determinant formula) in terms of the coefficients of 
    \(A\) that evaluates to \(0\). In others words, \(det(A) = 0 \iff\) we can construct a matrix \(B\) which, when multipled with \(A\), performs the corresponding arithmetic to return \(0\). Therefore, \(A\) is a zero divisor \(\iff det(A) = 0\).
\end{proof}

\end{homeworkProblem}

\begin{homeworkProblem}
    Let \(f: R\ \to S\) be a surjective ring homomorphism, $I\subset S$ an ideal. We need to show that $f^{-1}(I) \subset R$ is an ideal that contains $\ker(f)$.
    \begin{proof}
        From the correspondence theorem for groups, we know that \(f^{-1}(I)\) is a subgroup. We need to verify \(Rf^{-1}(I)\subset R\). Let \(x \in f^{-1}(I), r\in R\). We need to show that \(xr\in f^{-1}(I)\) i.e. \(f(xr) \in I\). This is immediate since \(f(xr) = f(x)f(r) \in I\) since \(f(x)\in I\). Also, since \(0\in I, f^{-1}(0) = \ker(f) \subset f^{-1}(I)\).  \\

        To complete the proof, we need to show that if \(I\subset R\) is an ideal containing \(\ker(f)\), then $f(I)\subset S$ is an ideal. Let $s\in S$. Since $f$ is surjective, we can write $s = f(r)$ for some \(r\in R\). Let \(y\in f(S)\). \(f(s)(y) = f(sy)\). Since \(sy \in I, f(sy) \in f(I)\). Therefore, \(f(I)\) is an ideal. This yields the required bijection.
    \end{proof}
\end{homeworkProblem}

\begin{homeworkProblem}
    \begin{proof}
        \begin{enumerate}
            \item[(a)] Let \(R\) be a commutative ring. Let \(a, b \in Nil(R)\). So, \(\exists n,n' \in \Z: a^n = b^{n'} = 0\). Let \(m = \mathrm{max}\left\{ n,n' \right\}\). Then \((a+b)^{2m} = a^{2m} + \ldots + a^{m}b^{m}+\ldots + b^{2m}= 0\) so \(a+b \in Nil(R)\).
            Let \(r\in R\). Then \((ra)^{n} = r^na^n = 0\) so \(ra\in Nil(R)\). Therefore, it it is a ring. \\
            Assume that \(\exists \; r + Nil(R) \in R/Nil(R)\) such that \((r+Nil(R))^m = r^m + Nil(R) = Nil(R)\) for some \(m \in \Z\). This implies that $r^m \in Nil(R)$ so $r \in Nil(R)$ and $r + Nil(R) = Nil(R) = O_{R/Nil(R)}$.
            \item[(b)] Let \(f(t) = a_0 + a_1t + \ldots + a_nt^n \in R[t]\). Assume that all \(a_i\) are nilpotent. 
            Since \(Nil(R[t])\) is a ring, \(a_i \in Nil(R[t]) \Rightarrow a_it_i \in Nil(R[t]) \) and \(a,b \in Nil(R[t]) \Rightarrow a+b\in Nil(R[t])\). 
            Therefore, \(f(t) \in Nil(R) \). Assume that \(f(t)\) is nilpotent i.e. \((a_0 + a_1t + \cdots + a_nt^n)^m = 0\) for some \(m\in \Z\). 
            In particular, \(a_0^m = 0\) so \(a_0\) is nilpotent. By closure of a ring, \(t(a_1 + a_2t + \cdots + a_nt^{n-1})\) is nilpotent. 
            So \(a_1\) is nilpotent as before.  We can proceed by induction to prove that all \(a_i\) must be nilpotent.
        \end{enumerate}
    \end{proof}
\end{homeworkProblem}
\begin{homeworkProblem}
    \begin{proof}

        \begin{enumerate}
            \item[(a)] Let \(a\in R\) be nilpotent i.e. \(\exists \in \N\) such \(a^n = 0\). We want to find some \(b \in R\) such that \((1+a)b = 1\). Recall that \((1+a)(1-a+\cdots + (-1)^{n-1}a^{n-1}) = 1 + (-1)^na^n = 1\). Therefore, \(1+a \in R^{\times}\)
            \item[(b)] Assume that \(a_0\) is invertible and \(a_i, i\geq 1\) are nilpotent. 
            \begin{lemma}
                Let \(u\in R^{\times}, a \in Nil(R)\). Then, \(u+a\in R^{\times}\).
            \end{lemma}
            \begin{proof}[Proof of lemma]
                We write \(u + a = u(1+u^{-1}a)\). Since \(u^{-1}a\) is nilpotent, \(1+u^{-1}a\) is invertible by (a) so \(u+a\) is invertible. 
            \end{proof}

            Appying this lemma to \(f(t) = a_0 + a_1t + \cdots + a_nt^n\), we are given that \(a_0\) is invertible and it is easy to see that
            \(a_1t + \cdots + a_nt^n\) is nilpotent. Therefore, \(f(t)\) is invertible.\\
            Conversely, assume that \(f(t)\) is invertible. Then \(\exists g(t) = b_0 + b_1t + \cdots + b_mt^m\)
            such that \(f(t)g(t) = 1\). Upon expanding and comparing terms, we see that \(a_0b_0 = 1\) so \(a_0\) is invertible. 
            Also, \(a_nb_m = 0\) and \(a_{n-1}b_m + a_nb_{m-1} = 0\). Multiplying across by \(a_n\), we get \(a_n^2b_{m-1} = 0\). We can keep repeating this step till it cascades down to \(a_n^mb_0 = 0\). Since \(b_0\) is a unit, \(a_n \in Nil(R)\). By induction, we get that all \(a_i \in Nil(R)\).
        \end{enumerate}
    \end{proof}

    
\end{homeworkProblem}

\begin{homeworkProblem}
\begin{proof}
        Let \(r\in R\) and \(f:\Z[t]\to R\) be a ring homomorphism such that
        \(f(t) = r\). Let \(p(t)\in \Z[t], p(t) = a_0 + a_1t + \cdots + a_nt^n, a_i \in \Z\). 
        Then, \(f(p) = f(a_0) + f(a_1)r + \cdots f(a_n)r^n\) which is unique since \(f(m)\) is uniquely determined for \(m \in \Z\).
        We know that \(Im(f)\subset R\) is a subring that contains \(r\). To prove that it is the smallest such subring, let \(S\subset R\) be a subring such that \(r\in S\). By closure under a ring, \(r^i \in S, a_ir_i \in S\) for \(a_i\in \Z\) and \(r^i + r^j \in S\). Therefore, \(Im(f) \in S\).  
\end{proof}
\end{homeworkProblem}

\begin{homeworkProblem}
\begin{proof}
        Let \(R\) be a domain such that \(R[t]\) is a PID. Recall that \(R \cong R[t]/(t)\). Since \(R\) is a domain, so \((t)\subset R[t]\) is a prime ideal. 
        Now we show that every prime ideal \(P\) in a PID is also maximal. \\
    
        Let \(P\) be a prime ideal and \(P \subset I \subset R\). Since \(R\) is a PID, we write \(P = (a), I = (b)\). Clearly \(a\in(b)\) therefore we can write \(a = bc\) for some \(c \in R\).
        Since \(P\) is prime, this means that \(b \in P\) or \(c \in P\). If \(b \in P\) then \(I = (b) \subset P\) and \(I=P\). If \(c \in P\) we can write \(c = ad\) for \(c\in R\). Then, \(a = bad\) i.e.
        \(bd = 1\) so \(b \in R^{\times}\). Then, \(I = R\).
        We have shown that \(P \subset I \subset R\) implies that \(I = P\) or \(I = R\). Therefore, \(P\) is maximal.\\
    
        We are almost done. From above, \((t)\) is maximal. Therefore, \(R \cong R[t]/(t)\) is a field. 
\end{proof}
\end{homeworkProblem}

\begin{homeworkProblem}
    \begin{proof}
        Let \(R\) be a non-zero commutative ring. By Zorn's lemma, it has a prime ideal \(P\). Now consider \(P \in R[t]\). \(xy \in P \Rightarrow x\in P \lor y\in P\) so \(P\) is prime in \(R[t]\) as well.
        It then follows that we can construct infinitely many more prime ideals \((Pt), (Pt^2), (Pt^3), \ldots\)
    \end{proof}
\end{homeworkProblem}

\begin{homeworkProblem}
    We define \(Rad(R) = \bigcap\limits_{M\subset I}M\).
    \begin{proof}
        \(\Leftarrow\) Assume that \(\forall y\in R, 1-xy \in R^{\times}\). 
        Also assume, in search of a contradiction, that \(x \notin Rad(R)\) i.e. \(\exists M: x \notin M\). Then, \((x) + M = R\). In particular, \(\exists y\in R, m \in M: xy + m = 1\).
        We can write \(m = 1 - xy\). By hypothesis, this is invertible. But then \(M\) contains an invertible element so \(M = R\) which is a contradiction. So, \(x\in Rad(R)\).\\
        \(\Rightarrow\) Assume that \(x \in Rad(R)\) and assume, in search of contradiction, that \(\exists y\in R: 1 - xy\) is not invertible. Any element that is not invertible is 
        contained in some maximal ideal so \(1-xy \in M\) for some \(M\). Write \(m = 1 -xy\). But since \(x \in Rad(R)\) by assumption, \(x \in M\) so \(xy \in M\). Then \(m+xy = 1 \in M\) and \(M = R\), a contradiction. 
        So \(\forall y\in R, 1-xy \in R^{\times}\).
    \end{proof}
\end{homeworkProblem}

\begin{homeworkProblem}
    \begin{proof}
        Let \(X\) be a set, \(R\) be a commutative ring. Recall that a ring homomorphism \(h: \Z[X] \to R\) is uniquely determined by a set map
        \(f: X \to R\). In other words, we have a bijection \(Hom_{CRings}(\Z(X), R \cong Maps(X,R)\). Let \(F\) be the forgetful functor and \(G\) be the functor that takes any set \(X \mapsto \Z[X]\). 
        Then, we can rewrite the above bijection as \(Hom_{CRings}(G(X), R \cong Maps(X,F(R)))\). Clearly, \(G\) is left-adjoint to \(F\).
    \end{proof}
\end{homeworkProblem}

\end{document}