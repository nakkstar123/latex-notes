\documentclass{article}

\usepackage{fancyhdr}
\usepackage{extramarks}
\usepackage{amsmath}
\usepackage{amsthm}
\usepackage{amsfonts}
\usepackage{amssymb}
\usepackage{tikz}
\usepackage[plain]{algorithm}
\usepackage{algpseudocode}

\newcommand\N{\ensuremath{\mathbb{N}}}
\newcommand\R{\ensuremath{\mathbb{R}}}
\newcommand\Z{\ensuremath{\mathbb{Z}}}
\renewcommand\O{\ensuremath{\emptyset}}
\newcommand\Q{\ensuremath{\mathbb{Q}}}
\newcommand\C{\ensuremath{\mathbb{C}}}
\newcommand\Ha{\ensuremath{\mathbb{H}}}
\newcommand\cont{\Rightarrow\!\Leftarrow}

\usetikzlibrary{automata,positioning}

\newtheorem{theorem}{Theorem}[section]
\newtheorem{corollary}{Corollary}[theorem]
\newtheorem{lemma}[]{Lemma}

%
% Basic Document Settings
%

\topmargin=-0.45in
\evensidemargin=0in
\oddsidemargin=0in
\textwidth=6.5in
\textheight=9.0in
\headsep=0.25in

\linespread{1.1}

\pagestyle{fancy}
\lhead{\hmwkAuthorName}
\chead{\hmwkClass\ : \hmwkTitle}
\rhead{\firstxmark}
\lfoot{\lastxmark}
\cfoot{\thepage}

\renewcommand\headrulewidth{0.4pt}
\renewcommand\footrulewidth{0.4pt}

\setlength\parindent{0pt}

%
% Create Problem Sections
%

\newcommand{\enterProblemHeader}[1]{
    \nobreak\extramarks{}{Problem \arabic{#1} continued on next page\ldots}\nobreak{}
    \nobreak\extramarks{Problem \arabic{#1} (continued)}{Problem \arabic{#1} continued on next page\ldots}\nobreak{}
}

\newcommand{\exitProblemHeader}[1]{
    \nobreak\extramarks{Problem \arabic{#1} (continued)}{Problem \arabic{#1} continued on next page\ldots}\nobreak{}
    \stepcounter{#1}
    \nobreak\extramarks{Problem \arabic{#1}}{}\nobreak{}
}

\setcounter{secnumdepth}{0}
\newcounter{partCounter}
\newcounter{homeworkProblemCounter}
\setcounter{homeworkProblemCounter}{1}
\nobreak\extramarks{Problem \arabic{homeworkProblemCounter}}{}\nobreak{}

%
% Homework Problem Environment
%
% This environment takes an optional argument. When given, it will adjust the
% problem counter. This is useful for when the problems given for your
% assignment aren't sequential. See the last 3 problems of this template for an
% example.
%
\newenvironment{homeworkProblem}[1][-1]{
    \ifnum#1>0
        \setcounter{homeworkProblemCounter}{#1}
    \fi
    \section{Problem \arabic{homeworkProblemCounter}}
    \setcounter{partCounter}{1}
    \enterProblemHeader{homeworkProblemCounter}
}{
    \exitProblemHeader{homeworkProblemCounter}
}

%
% Homework Details
%   - Title
%   - Due date
%   - Class
%   - Section/Time
%   - Instructor
%   - Author
%

\newcommand{\hmwkTitle}{Homework\ \#3}
\newcommand{\hmwkDueDate}{February 2, 2023}
\newcommand{\hmwkClass}{Math 210B}
\newcommand{\hmwkClassTime}{210B}
\newcommand{\hmwkClassInstructor}{Professor Alexander Merkurjev}
\newcommand{\hmwkAuthorName}{\textbf{Nakul Khambhati}}

%
% Title Page
%

\title{
    \vspace{2in}
    \textmd{\textbf{\hmwkClass:\ \hmwkTitle}}\\
    \normalsize\vspace{0.1in}\small{Due\ on\ \hmwkDueDate}\\
    \vspace{0.1in}\large{\textit{\hmwkClassInstructor}}
    \vspace{3in}
}

\author{\hmwkAuthorName}
\date{}

\renewcommand{\part}[1]{\textbf{\large Part \Alph{partCounter}}\stepcounter{partCounter}\\}

%
% Various Helper Commands
%

% Useful for algorithms
\newcommand{\alg}[1]{\textsc{\bfseries \footnotesize #1}}

% For derivatives
\newcommand{\deriv}[1]{\frac{\mathrm{d}}{\mathrm{d}x} (#1)}

% For partial derivatives
\newcommand{\pderiv}[2]{\frac{\partial}{\partial #1} (#2)}

% Integral dx
\newcommand{\dx}{\mathrm{d}x}

% Alias for the Solution section header
\newcommand{\solution}{\textbf{\large Solution}}

% Probability commands: Expectation, Variance, Covariance, Bias
\newcommand{\E}{\mathrm{E}}
\newcommand{\Var}{\mathrm{Var}}
\newcommand{\Cov}{\mathrm{Cov}}
\newcommand{\Bias}{\mathrm{Bias}}

\begin{document}

\maketitle

\pagebreak

\begin{homeworkProblem}
    We are asked to show that \(\Z[\sqrt{2}] = \left\{ a + b\sqrt{2}, a,b \in \Z \right\}\) is Euclidean. We claim that \(\phi(a+b\sqrt{2}) = 
    |a^2 - 2b^2|\) is a Euclidean function for this ring. Let \(\alpha = a_1+a_2\sqrt{2}, \beta = b_1 + b_2\sqrt{2} \in \Z[\sqrt{2}]\). We need to show that we can divide
    with remainder using this formula. In other words, we can write \(\alpha = \gamma \beta + \delta\) for some \(\gamma, \delta \in \Z[\sqrt{2}]\) with
    \(\phi(\delta) < \phi(\beta)\). Note that this division is always possible in \(\Q[\sqrt{2}]\) as \(\dfrac{(a_1b_1 - 2a_2b_2) + (b_1a_2 - a_1b_2)\sqrt{2}}{b_1^2 - 2b_2^2}\)
    which we can write as \(c_1 + c_2\sqrt{2}\) where \(c_1, c_2\) can be read off the previous equation. In general, \(c_1, c_2 \in \Q \). However, we can pick the nearest integers
    \(q_1, q_2 \in \Z\) such that \(|c_1 - q_1| \leq \frac{1}{2}\) and \(|c_2 - q_2| \leq \frac{1}{2}\). Set \(\gamma = q_1 + q_2\sqrt{2}\). Next, set \(\theta = \frac{\alpha}{\beta} - \gamma\) 
    so that \(\theta \beta = \alpha - \gamma \beta\). Then, setting \(\delta = \theta \beta\) we have \(\alpha = \gamma\beta + \delta\). It remains to show that \(\phi(\delta) < \phi(\beta)\).
    So, it suffices to show that \(\phi(\theta) < 1\). We can evaluate \(\phi(\theta) \leq (c_1 - q_1)^2 + 2(c_2 - q_2)^2 \leq \frac{3}{4}\).
\end{homeworkProblem}

\begin{homeworkProblem}
    We already saw that \(\Z[\sqrt{-5}]\) is not a principal ideal. We now need to show some ideal in the ring that cannot be expressed as \(aR\) for any \(a\). Recall that since it is a principal ideal, 
    there is some Euclidean norm \(N(r)\) on the ring. It can be checked that \(N(z) = a^2 + 5b^2\) is a valid norm for \(z = a + b\sqrt{5}\). Consider the ideal \(I = \langle 3, 1 + \sqrt{-5} \rangle\).
    If the ideal is principal then \(\exists z: I = \langle z \rangle\). Then by properties of \(N\), we have that \(N(z)| 9\) and \(N(z)|6\) so \(N(z)|3\). In particular, \(z = a + b\sqrt{-5}\) has \( b = 0\)
     so \(a = \pm 1\). But then \(I = R\) which is a contradiction. 
\end{homeworkProblem}

\begin{homeworkProblem}
    Here, we will use the fact that a prime \(p\) is the sum of two squares \(\iff\) \(p = 1\) (mod 4). Therefore, 
    if \(p = 3\) (mod 4) then we cannot write it as the sum of squares. So, assume that \(p\) has a non-trivial factorization
    \(p = mn\) so that \(N(p) = p^2 = N(m)N(n)\). So, we must have \(N(m) = p, N(n) = p\). In particular, we have written \(p\) as 
    the sum of squares \(a^2 + b^2\). This is a contradiction. So every factorization is trivial. So \(p\) is irreducible. In a PID, this means \(p\)
    is prime in \(\Z[i]\). \(2 = (1-i)(1+i)\) is not prime. 
\end{homeworkProblem}

\begin{homeworkProblem}
    Assume that \(p = 1\) (mod 4). We then know that \(x^2 = -1\) (mod p) for some \(x \in \Z/p\Z\). As a result, \(p|(x^2+1)\)
    so \(p|(x+i)(x-i)\). But since \(p\) does not divide either of the factors, \(p\) is not a prime in \(\Z[i]\). Then, there is
    a nontrivial factorization \( p = z_1z_2\). So then \(z_1 \in \Z[i]\setminus\Z\) and \(z_1 = a^2 + b^2\) for nonzero \(a,b\). 
    So, it is the sum of squares. 
\end{homeworkProblem}

\begin{homeworkProblem}
    Consider the prime factorization of \(10 = 2\cdot 5 = (1+i)(1-i)(2+i)(2-i)\). The prime ideals in \(\Z[i]\) are \((1+i),(p)\) 
    for \(p = 3\) (mod 4). So, \((i+1), (2+i), (2-i)\) are prime ideals that contain \(10\). There are a total of \(3\). 
\end{homeworkProblem}

\begin{homeworkProblem}
    Assume that \(p\) is a prime integer. Also assume that it has two distinct representations as sums of squares. 
    Now, extend to \(\Z[i]\). Assume that \(p\) is a prime gaussian integer. But then \(p^2 = (a^2 + b^2)(c^2+d^2)\) so that \(N(z) = N(z_1)N(z_2)\)
    for some gaussian integers \(z_1, z_2\). This shows that \(p\) is not prime in \(\Z[i]\). However, \(p\) was taken as an integer so \(p=1\) (mod 4). However, 
    since we are given two distinct representations of squares that sum up to \(p\) we cannot have \(p = 1\) (mod 4). Therefore, we have a contradiction. So \(p\)
    is not a prime integer. 
\end{homeworkProblem}
    
\begin{homeworkProblem}
    The proof follows from degree considerations. First, note that \(R\subset R[x]\) and that units in \(R[x]\) are precisely the units in \(R\). Since \(R[x]\) is a UFD,
    therefore a domain, even \(R\) is a domain. We know that for any \(x \in R\), we have a unique factorization in \(R[x]\) so we can write \(x = p_1\cdots p_n\) where each \(p_i\)
    is prime (i.e. irreducible) in \(R[x]\). It suffices to show that each \(p_i\) is in \(R\). If it's irreducible in \(R[x]\) then it is definitely irreducible in the subring \(R\).
    Use the fact that \(\deg(fg) = \deg(f)\deg(g)\). Since \(\deg(x) = 0\) we must have \(\deg(p_i)=0\) so each \(p_i \in R\).
\end{homeworkProblem}

\begin{homeworkProblem}
    We are asked to show that \(p = x^9 + y^9 + z^9\) is irreducible in \(\C[x,y,z]\). We treat this ring as \(\C[x,y][z]\). Consider \(q = x+y\) which is irreducible therefore prime. It is clear that
    \(q|(x^9 + y^9)\) and \(q \nmid z^9\). Further, \(q^2 \nmid (x^9 + y^9)\). This satisfies the Eisenstein criterion so \(p\) is irreducible in \(\C[x,y,z]\).
\end{homeworkProblem}

\begin{homeworkProblem}
    Let \(f,g \in R[x]\) such that \(C(g) = R\) and \(f = gh\) for some \(h \in F[x]\). First note that we can multiply \(h\) by some \(a \in R\) to get \(ah \in R[x]\).
    Let's denote \(C(ah) = bR\). Then, \(af = g(ah)\). So, \(a(C(f)) = bR\). Then, \(a|b\) so that \(b = ac\) for some \(c \in R\). Then, \(C(ah) = acR\) or \(ac\) divides every coefficient of \(ah\).
    So \(c\) divides every coefficient of \(h\) and \(h \in R[x]\).
\end{homeworkProblem}

\begin{homeworkProblem}
    In class we proved that if \(R\) is a UFD, then \(R[x]\) is a UFD. Since \(\Q\) is a field so a PID, it is clearly a UFD. 
    Therefore, by induction, each \(\Q[x_1, \ldots, x_n]\) is a UFD, so even \(\Q[x_1, x_2, \ldots]\) is a UFD. However, this ring is clearly 
    not finitely generated since \((1, x_1, x_2, \ldots)\) is the smallest generating set in the ring. Therefore, it is not Noetherian. 
\end{homeworkProblem}

\end{document}