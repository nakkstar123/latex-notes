\documentclass{article}

\usepackage{fancyhdr}
\usepackage{extramarks}
\usepackage{amsmath}
\usepackage{amsthm}
\usepackage{amsfonts}
\usepackage{tikz}
\usepackage[plain]{algorithm}
\usepackage{algpseudocode}

\newcommand\N{\ensuremath{\mathbb{N}}}
\newcommand\R{\ensuremath{\mathbb{R}}}
\newcommand\Z{\ensuremath{\mathbb{Z}}}
\renewcommand\O{\ensuremath{\emptyset}}
\newcommand\Q{\ensuremath{\mathbb{Q}}}
\newcommand\C{\ensuremath{\mathbb{C}}}
\newcommand\Ha{\ensuremath{\mathbb{H}}}
\newcommand\cont{\Rightarrow\!\Leftarrow}


% macros for 245A
\newcommand\A{\mathcal{A}}
\newcommand\F{\mathcal{F}}

\usetikzlibrary{automata,positioning}

\newtheorem{theorem}{Theorem}[section]
\newtheorem{corollary}{Corollary}[theorem]
\newtheorem{lemma}[]{Lemma}

%
% Basic Document Settings
%

\topmargin=-0.45in
\evensidemargin=0in
\oddsidemargin=0in
\textwidth=6.5in
\textheight=9.0in
\headsep=0.25in

\linespread{1.1}

\pagestyle{fancy}
\lhead{\hmwkAuthorName}
\chead{\hmwkClass\ : \hmwkTitle}
\rhead{\firstxmark}
\lfoot{\lastxmark}
\cfoot{\thepage}

\renewcommand\headrulewidth{0.4pt}
\renewcommand\footrulewidth{0.4pt}

\setlength\parindent{0pt}

%
% Create Problem Sections
%

\newcommand{\enterProblemHeader}[1]{
    \nobreak\extramarks{}{Problem \arabic{#1} continued on next page\ldots}\nobreak{}
    \nobreak\extramarks{Problem \arabic{#1} (continued)}{Problem \arabic{#1} continued on next page\ldots}\nobreak{}
}

\newcommand{\exitProblemHeader}[1]{
    \nobreak\extramarks{Problem \arabic{#1} (continued)}{Problem \arabic{#1} continued on next page\ldots}\nobreak{}
    \stepcounter{#1}
    \nobreak\extramarks{Problem \arabic{#1}}{}\nobreak{}
}

\setcounter{secnumdepth}{0}
\newcounter{partCounter}
\newcounter{homeworkProblemCounter}
\setcounter{homeworkProblemCounter}{1}
\nobreak\extramarks{Problem \arabic{homeworkProblemCounter}}{}\nobreak{}

%
% Homework Problem Environment
%
% This environment takes an optional argument. When given, it will adjust the
% problem counter. This is useful for when the problems given for your
% assignment aren't sequential. See the last 3 problems of this template for an
% example.
%
\newenvironment{homeworkProblem}[1][-1]{
    \ifnum#1>0
        \setcounter{homeworkProblemCounter}{#1}
    \fi
    \section{Problem \arabic{homeworkProblemCounter}}
    \setcounter{partCounter}{1}
    \enterProblemHeader{homeworkProblemCounter}
}{
    \exitProblemHeader{homeworkProblemCounter}
}

%
% Homework Details
%   - Title
%   - Due date
%   - Class
%   - Section/Time
%   - Instructor
%   - Author
%

\newcommand{\hmwkTitle}{Homework\ \#1}
\newcommand{\hmwkDueDate}{October 6, 2023}
\newcommand{\hmwkClass}{Math 245A - Real Analysis}
\newcommand{\hmwkClassTime}{245A}
\newcommand{\hmwkClassInstructor}{Professor Mario Bonk}
\newcommand{\hmwkAuthorName}{\textbf{Nakul Khambhati}}

%
% Title Page
%

\title{
    \vspace{2in}
    \textmd{\textbf{\hmwkClass:\ \hmwkTitle}}\\
    \normalsize\vspace{0.1in}\small{Due\ on\ \hmwkDueDate}\\
    \vspace{0.1in}\large{\textit{\hmwkClassInstructor}}
    \vspace{3in}
}

\author{\hmwkAuthorName}
\date{}

\renewcommand{\part}[1]{\textbf{\large Part \Alph{partCounter}}\stepcounter{partCounter}\\}

%
% Various Helper Commands
%

% Useful for algorithms
\newcommand{\alg}[1]{\textsc{\bfseries \footnotesize #1}}

% For derivatives
\newcommand{\deriv}[1]{\frac{\mathrm{d}}{\mathrm{d}x} (#1)}

% For partial derivatives
\newcommand{\pderiv}[2]{\frac{\partial}{\partial #1} (#2)}

% Integral dx
\newcommand{\dx}{\mathrm{d}x}

% Alias for the Solution section header
\newcommand{\solution}{\textbf{\large Solution}}

% Probability commands: Expectation, Variance, Covariance, Bias
\newcommand{\E}{\mathrm{E}}
\newcommand{\Var}{\mathrm{Var}}
\newcommand{\Cov}{\mathrm{Cov}}
\newcommand{\Bias}{\mathrm{Bias}}

\begin{document}

\maketitle

\pagebreak



\begin{homeworkProblem}
    We proceed by verifying the 3 conditions for $\A$ to be an algebra:
	\begin{enumerate}
		\item $\O \in \F \subset \A$. This is sufficient to show $X \in \A$ once we show it is closed under taking complement.
		\item Let $A \in \A$ so we can write $A = A_1 \cup \cdots \cup A_m$ where $A_i \in \F$. Then by condition (iii) for each $i \in [m]$ there exists  $n_i \in \N$ and pairwise disjoint $A_{i1},\ldots, A_{in_i} \in \F$ such that $A_i^c = A_{i1} \cup \ldots \cup A_{in_i}$. So, $A^c = A_1^c \cap \cdots \cap A_m^c  = \bigcup_{j_1 \in [n_1], \ldots, j_m \in [n_m]} (A_{1j_1} \cap \cdots \cap A_{mj_m}) \in \A$ since we are taking a disjoint union of elements in $\F$ and the family is closed under finite intersection.
		\item Let  $A,B \in \A$. It suffices to show that $A \cap B \in \A$ as then $A\cup B = \left( A^c \cap B^c \right)^c$ and we have already checked that $\A$ is closed under taking complement. By assumption, we can write $A = A_1 \cup \ldots \cup A_n$ and $B = B_1 \cup \ldots \cup B_m$ where $A_i, B_j \in \F$. Then $A \cap B = \bigcup_{i,j = 1}^n \left( A_i \cap B_j \right) \in \A$ since $\F$ is closed under intersections so we have expressed  $A \cap B$ as a finite disjoint union of sets in $\F$. 
	\end{enumerate}
\end{homeworkProblem}

\begin{homeworkProblem}
	\begin{enumerate}
		\item[(a)] First we verify the conditions for $\A$ to be an algebra on  $X$. 
		 \begin{enumerate}
			 \item[1.] By considering $I = \O$ and $I = [n]$ we see that  $ \O = \bigcup_{i \in \O} M_i \in \A$ and $ X = \bigcup_{i \in [n]} M_i \in \A$.
			 \item[2.] Let $M \in \A$ so we can write $M = \bigcup_{i \in I} M_i $ for some $I \subset [n]$. Since the $\{M_i\}_{i \in [n]}$ form a partition of  $X$, we can write $M^c = (\bigcup_{i \in I} M_i)^c = \bigcup_{i \in I^c} M_i \in \A$. 
			 \item[3.] Let $M_1 = \bigcup_{i \in  I_1}M_i, M_2 = \bigcup_{i \in I_2}M_i$ be elements of $\A$. Then, $M_1 \cup M_2 = \bigcup_{i \in  I_1\cup I_2} M_i \in \A$.   
		\end{enumerate}
		Elements of $\A$ are in bijection with subsets of  $[n]$ so there are $2^n$ elements in the algebra. Yes, every finite algebra is a $\sigma$-algebra since countable unions and finite unions are the same.
	\item[(b)] \textit{Need to explicitly construct this set somehow by partitioning based on disjoint parts. Just need to formalize this construction.}
	\end{enumerate}
\end{homeworkProblem}

\begin{homeworkProblem}
	
\end{homeworkProblem}

\begin{homeworkProblem}
	
\end{homeworkProblem}

\begin{homeworkProblem}
	\begin{enumerate}
		\item[(a)] We are given that $\A$ is an algebra on  $X$. First assume that it is a $\sigma$-algebra and  $A_n \nearrow$. We know that a  $\sigma$-algebra is closed under countable unions so  $\bigcup_{n \in \N} A_n \in \A$. Conversely assume that the property holds and we need to show that $\A$ is closed under countable unions. Let  $\{B_n\}_{n \in \N}$ be an arbitrary collection of elements in $\A$. Define  $A_n = \bigcup_{i \in  [n]} B_i$ so it is the union of the first $n$ elements in the collection. Since $\A$ is an algebra, each  $B_i \in \A$ by closure under finite unions. Clearly  $A_n \subset A_{n+1}$ so $A_n \nearrow$. But also note that for all  $n$ we have  $\bigcup_{i \in  [n]} A_i = A_n = \bigcup_{i \in  [n]} B_i$ so in particular $\bigcup_{n \in \N} B_n = \bigcup_{n \in \N} A_n \in \A$ by the property we assumed and so we are done. 
		\item[(b)] We need to show that $\mu$ has countable additivity assuming it has finite additivity and the property stated. Let $\{B_n\}_{n\in\N} \in \A$ be a collection of pairwise disjoint sets. We define $A_n = \bigcup_{i \in [n]}B_i$ as above so that $A_n \nearrow$. Then $\mu\left( \bigcup_{n \in  \N}B_n  \right) = \mu\left( \bigcup_{n \in  \N}A_n  \right) = \lim_{N \to \infty} \mu\left(A_n  \right)$. But from finite additivity, we can write $\mu\left( A_n \right) = \sum_{i=1}^{n} \mu\left( B_i \right)$ so the term above simplifies to $\sum_{n=1}^{\infty} \mu\left( B_n \right)$. 
	\end{enumerate}
	
\end{homeworkProblem}

\begin{homeworkProblem}
	\begin{enumerate}
		\item[(a)] Note that $x \in  A$ if and only if for all $n \in \N$, there exists some  $m \geq n$ such that  $x \in A_m$. This is because if $x \in A$ i.e. it is in infinitely many $A_m$ then it is also in infinitely many  $A_m$ if we exclude a finite number of sets (say the first $n$ sets). On the other hand, if it is not in infinitely many  $A_n$ there is an $N$ such that for all for all $n > N$,  $ x \notin A_n$ and so the right hand side becomes false. Finally, we can write $\forall n \in \N, \exists m \geq n: x \in  A_m \iff x \in \bigcap_{n \in \N} \bigcup_{m\geq n} A_m$. $A$ is a countable intersection of a countable union of sets in  $\A$ and is therefore in  $\A$.
		\item[(b)] Let $B_n$ denote $\bigcup_{m\geq n}A_m$. Then $A = \bigcap_{n \in \N}B_n$ and $B_n \searrow$. Also  $\mu\left( B_1 \right)  = \mu\left( \bigcup_{n \in \N}A_n \right) \leq \sum\limits_{n=1}^{\infty}\mu\left( A_n \right) < \infty$ so we can apply continuity from above to get $\mu(A) = \lim_{n \to  \infty} \mu\left( B_n \right)$. Then, by countable subadditivity, $\mu(B_n) \leq \sum_{m=n}^{\infty} \mu\left( A_m \right)$ which goes to $0$ as  $n\to \infty$ because $\sum\limits_{n=1}^{\infty}\mu\left( A_n \right) < \infty$. 
	\end{enumerate}
	\end{homeworkProblem}

\begin{homeworkProblem}

	\begin{enumerate}
		\item[(a)] Pick $A_1$ to be a set such that there are infinitely many sets in $\A$ that intersect with  $A_1^c$. Such an $A_1$ always exists because  $\A$ has infinite elements. Next, choose $A_2$ from $\{A_1^c \cap B : B \in \A\}$ such that there are infinitely many sets in $\A$ that intersect with both $A_1^c$ and  $A_2^c$. This way, $A_1$ and  $A_2$ are disjoint. We can repeat this process indefinitely. Formally, assume we have picked  $A_1, A_2, \ldots, A_{n-1}$ that are disjoint and there are infinite sets in $\A$ that intersect with all $A_i^c$. Then, once again, we can pick $A_n$ disjoint from the others such that infinitely many sets in  $\A$ intersect it. This gives us an infinite series of disjoint sets in  $\A$.  
		\item[(b)] Call this disjoint family $\F = \{A_i\}_{i \in \N}$. Now consider the collection $\mathcal{G}$ of all sets that can be obtained by taking (disjoint) unions of sets in this family. Each element in $\mathcal{G}$ is also in $\A$ since it is closed under countable unions. Since the  $A_i$ are disjoint, this collection $\mathcal{G}$ is in bijection with subsets of  $\N$ which we know is uncountable by Cantor's diagonal argument. Since $\mathcal{G} \subset \A$, we also get that $\A$ is uncountable.  
	\end{enumerate}
	
\end{homeworkProblem}



\end{document}
