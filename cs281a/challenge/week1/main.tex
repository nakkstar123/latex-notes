\documentclass{article}

\usepackage{fancyhdr}
\usepackage{extramarks}
\usepackage{amsmath, amssymb}
\usepackage{amsthm}
\usepackage{amsfonts}
\usepackage{tikz}
\usepackage[plain]{algorithm}
\usepackage{algpseudocode}

\newcommand\N{\ensuremath{\mathbb{N}}}
\newcommand\R{\ensuremath{\mathbb{R}}}
\newcommand\Z{\ensuremath{\mathbb{Z}}}
\renewcommand\O{\ensuremath{\emptyset}}
\newcommand\Q{\ensuremath{\mathbb{Q}}}
\newcommand\C{\ensuremath{\mathbb{C}}}
\newcommand\Ha{\ensuremath{\mathbb{H}}}
\newcommand\cont{\Rightarrow\!\Leftarrow}
\newcommand\tf{\therefore}

\usetikzlibrary{automata,positioning}

%
% Basic Document Settings
%

\topmargin=-0.45in
\evensidemargin=0in
\oddsidemargin=0in
\textwidth=6.5in
\textheight=9.0in
\headsep=0.25in

\linespread{1.1}

\pagestyle{fancy}
\lhead{\hmwkAuthorName}
\chead{\hmwkClass\ : \hmwkTitle}
\rhead{\firstxmark}
\lfoot{\lastxmark}
\cfoot{\thepage}

\renewcommand\headrulewidth{0.4pt}
\renewcommand\footrulewidth{0.4pt}

\setlength\parindent{0pt}

%
% Create Problem Sections
%

\newcommand{\enterProblemHeader}[1]{
    \nobreak\extramarks{}{Problem \arabic{#1} continued on next page\ldots}\nobreak{}
    \nobreak\extramarks{Problem \arabic{#1} (continued)}{Problem \arabic{#1} continued on next page\ldots}\nobreak{}
}

\newcommand{\exitProblemHeader}[1]{
    \nobreak\extramarks{Problem \arabic{#1} (continued)}{Problem \arabic{#1} continued on next page\ldots}\nobreak{}
    \stepcounter{#1}
    \nobreak\extramarks{Problem \arabic{#1}}{}\nobreak{}
}

\setcounter{secnumdepth}{0}
\newcounter{partCounter}
\newcounter{homeworkProblemCounter}
\setcounter{homeworkProblemCounter}{1}
\nobreak\extramarks{Problem \arabic{homeworkProblemCounter}}{}\nobreak{}

%
% Homework Problem Environment
%
% This environment takes an optional argument. When given, it will adjust the
% problem counter. This is useful for when the problems given for your
% assignment aren't sequential. See the last 3 problems of this template for an
% example.
%
\newenvironment{homeworkProblem}[1][-1]{
    \ifnum#1>0
        \setcounter{homeworkProblemCounter}{#1}
    \fi
    \section{Problem \arabic{homeworkProblemCounter}}
    \setcounter{partCounter}{1}
    \enterProblemHeader{homeworkProblemCounter}
}{
    \exitProblemHeader{homeworkProblemCounter}
}

%
% Homework Details
%   - Title
%   - Due date
%   - Class
%   - Section/Time
%   - Instructor
%   - Author
%

\newcommand{\hmwkTitle}{Challenge\ \#1}
\newcommand{\hmwkDueDate}{January 18, 2022}
\newcommand{\hmwkClass}{CS 281}
\newcommand{\hmwkClassTime}{131AH}
\newcommand{\hmwkClassInstructor}{Professor Alexander Sherstov}
\newcommand{\hmwkAuthorName}{\textbf{Nakul Khambhati}}

%
% Title Page
%

\title{
    \vspace{2in}
    \textmd{\textbf{\hmwkClass:\ \hmwkTitle}}\\
    % \normalsize\vspace{0.1in}\small{Due\ on\ \hmwkDueDate}\\
    \vspace{0.1in}\large{\textit{\hmwkClassInstructor}}
    \vspace{3in}
}

\author{\hmwkAuthorName}
\date{}

\renewcommand{\part}[1]{\textbf{\large Part \Alph{partCounter}}\stepcounter{partCounter}\\}

%
% Various Helper Commands
%

% Useful for algorithms
\newcommand{\alg}[1]{\textsc{\bfseries \footnotesize #1}}

% For derivatives
\newcommand{\deriv}[1]{\frac{\mathrm{d}}{\mathrm{d}x} (#1)}

% For partial derivatives
\newcommand{\pderiv}[2]{\frac{\partial}{\partial #1} (#2)}

% Integral dx
\newcommand{\dx}{\mathrm{d}x}

% Alias for the Solution section header
\newcommand{\solution}{\textbf{\large Solution}}

% Probability commands: Expectation, Variance, Covariance, Bias
\newcommand{\E}{\mathrm{E}}
\newcommand{\Var}{\mathrm{Var}}
\newcommand{\Cov}{\mathrm{Cov}}
\newcommand{\Bias}{\mathrm{Bias}}

\begin{document}

\maketitle

\pagebreak

\begin{homeworkProblem}
    While constructing the polynomial to compute \(\texttt{MOD}_{m}\),
    we raise the polynomial to the \(m\)-th degree so that every non-zero element
    evaluates to \(1\). If \(m\) was not a prime number, this would not work for all non-zero elements. 
    It would only work for the ones that are relatively prime to \(m\).
\end{homeworkProblem}

\begin{homeworkProblem}
    
\end{homeworkProblem}

\begin{homeworkProblem}
    We can prove that there is no polynomial-size circuit of constant depth that computes the majority function on \(n\) bits by proving the following proposition: \\
     \textit{\(\exists\) a poly-size const depth circuit that computes \texttt{MAJORITY} \(\implies\) \(\exists\) exists a poly-size const depth circuit that computes \texttt{PARITY}} \\
     Then the result follows by contraposition and Razborov-Smolensky. 
     \begin{proof}
        Assume that \(C\) is a poly-size const depth circuit that computes \texttt{MAJORITY}. Consider the problem 
        of determining whether \(\# x = k\) where \(\#x\) denotes the number of \(1\)'s in \(x\) and \(0\leq k \leq n\). We will now construct another circuit \(C'\) that is built out of \(C\) to solve this problem. \\

        We can use \(C\) to decide \(\#x\geq k\). This part is easier to explain by introducing some numbers. Assume \(n = 5, x = 11010\) such that \(\#x = 3\). 
        Suppose we want to check whether \(\#x\geq 4\) i.e. the case \(k = 4\). We can simply append \(00\) to \(x\) and compute \(C(x00) = 1 \iff \#x \geq 4\) since \(|x00| =7\). 
        This way, we can compute any \(\#x\leq k\) by appending a suitable number of \(1\)'s and \(0\)'s. \\
        Similarly, we can compute \(\#x \leq k\). For example, let's decide \(\#x \leq 2\). It can be verified that \(C(x) = 0 \iff \#x \leq 2 \).\\
        Then we can construct \(C'\) as \(\#x = k \iff \#x \leq k \land \#x \geq k\). Since \(C'\) depends on \(k\), let's denote it \(C_{k}'\)\\

        Now, we can easily compute \(\texttt{PARITY} = \bigvee\limits_{k \text{ odd}} C_k'\) since \texttt{PARITY} returns \texttt{TRUE} if and only if there is an odd number of \(1\)'s in \(x\). This only increase the depth by a constant factor and the size remains polynomial so we have proved the proposition.
     \end{proof}
\end{homeworkProblem}

\begin{homeworkProblem}
    
\end{homeworkProblem}


\end{document}